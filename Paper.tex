\documentclass[11pt]{article}
\IfFileExists{VERSION.tex}{\input{VERSION}}%
{\date{Pdf compiled \today. (Version unclear; run `make VERSION.tex'.)}}
\usepackage{%
  amsfonts,
  amsmath,
  amssymb,
  amsthm,
  booktabs,
  geometry,
  graphicx,
  setspace,
  slantsc,
  tabularx,
  url,
}

\usepackage[small]{caption}
\usepackage[mmddyyyy,hhmmss]{datetime}
\usepackage[T1]{fontenc}
\usepackage[multiple]{footmisc}
\usepackage[sort,round]{natbib}

\urlstyle{same}

\date{6 October, 2014}

\bibliographystyle{abbrvnat}
\newcommand\citepos[2][]{\citeauthor{#2}'s \citeyearpar[#1]{#2}}

\newtheorem{innercustomlem}{Lemma}
\newenvironment{customlem}[1]
  {\renewcommand\theinnercustomlem{#1}\innercustomlem}
  {\endinnercustomlem}

\newtheorem{thm}{Theorem}
\newtheorem{lem}{Lemma}
\newtheorem{claim}{Claim}
\newtheorem{cor}{Corollary}
\newtheorem{res}{Result}

\theoremstyle{definition}

\newtheorem{example}{Example}
\newtheorem{defn}{Definition}
\newtheorem{rem}{Remark}

\DeclareMathOperator*{\argmin}{arg\,min}
\DeclareMathOperator*{\plim}{plim}

\DeclareMathOperator{\dist}{d}
\DeclareMathOperator{\const}{const}
\DeclareMathOperator{\cov}{cov}
\DeclareMathOperator{\E}{E}
\DeclareMathOperator{\eqd}{\overset{d}{=}}
\DeclareMathOperator{\ind}{1}
\DeclareMathOperator{\pr}{Pr}
\DeclareMathOperator{\sgn}{sgn}
\DeclareMathOperator{\var}{var}
\DeclareMathOperator{\vech}{vech}

\renewcommand{\mod}{\operatorname{mod}}

\newcommand{\clt}{\textsc{clt}}
\newcommand{\fclt}{\textsc{fclt}}
\newcommand{\hac}{\textsc{hac}}
\newcommand{\lln}{\textsc{lln}}
\newcommand{\mds}{\textsc{mds}}
\newcommand{\ned}{\textsc{ned}}

\newcommand{\textif}{\text{if}}
\newcommand{\textand}{\text{and}}

\newcommand{\sigmafield}{$\sigma$-field}

\newcommand{\Fs}{\mathcal{F}}
\newcommand{\Gs}{\mathcal{G}}
\newcommand{\Ms}{\mathcal{M}_n}
\newcommand{\Pm}{\pr_{\mathcal{M}}}
\newcommand{\Em}{\E_{\mathcal{M}}}
\newcommand{\BigC}{~\Big|~}
\newcommand{\Zs}{\mathcal{Z}}
\DeclareMathOperator{\discreteuniform}{discrete\ uniform}

\frenchspacing

\input{VERSION}
\begin{document}

\author{Gray Calhoun\thanks{Economics Department, Iowa State
    University, Ames, IA 50011. Telephone: (515) 294-6271.  Email:
    \protect\url{gcalhoun@iastate.edu}. Web:
    \protect\url{http://gray.clhn.org}.  I would like to thank
    Helle Bunzel, Dimitris Politis, Robert Taylor, and three anonymous
    referees for their comments and feedback on earlier versions of
    this paper.}}

\title{Block bootstrap consistency\\under weak assumptions}

\maketitle

\begin{abstract}\noindent
  This paper weakens the size and moment conditions needed for typical
  block bootstrap methods (i.e.\ the Moving Blocks, Circular Blocks,
  and Stationary Bootstraps) to be valid for the sample mean of
  Near-Epoch-Dependent (\ned) functions of mixing processes; they are
  consistent under the weakest conditions that ensure the original
  \ned\ process obeys a Central Limit Theorem \citep[those
    of][\textit{Econometric Theory}]{Jon:97}.  In doing so, this paper
  extends De Jong's method of proof, a blocking argument, to hold with
  random and unequal block lengths.  This paper also proves that
  bootstrapped partial sums satisfy a Functional \clt\ under the same
  conditions.

  \noindent \allcaps{jel} Classification: C12, C15

  \noindent Keywords: Resampling, Time Series, Near Epoch Dependence,
  Functional Central Limit Theorem
\end{abstract}

\newpage
\noindent Block bootstraps, e.g.\ the Moving Blocks
(\citealp{Kun:89}, and \citealp{LiS:92}), Circular Block \citep{PoR:92}, and Stationary
Bootstraps \citep{PoR:94}, have become popular in Economics, partly
because they do not require the researcher to make parametric
assumptions about the data generating process.  They are valid under
general weak dependence and moment conditions.  Some recent papers
(\citealp{GoW:02}, and \citealp{GoJ:03}) relax the dependence and moment
conditions of the original papers to fit with the Near-Epoch-Dependence
(\ned) assumptions commonly used in econometrics.%
\footnote{\citet{GoW:02} show that these bootstrap methods can be
  applied to heterogeneous $L_{2+\delta}$-\ned\ processes of size
  $-2(r-1)/(r-2)$ on a strong mixing sequence of size
  $-r(2+\delta)/(r-2)$, where $r > 2$ and $\delta >0$, when the
  original series has uniformly bounded $3r$-moments.  \citet{GoJ:03}
  relax these conditions to $L_{2+\delta}$-\ned\ of size $-1$ and
  $r+\delta$ moments for the original series, and size
  $-(2+\delta)(r+\delta)/(r-2)$ for the underlying mixing series.
  Both papers require that the expected block length grow with $n$ and
  be $o(n^{1/2})$.  \cite{GoP:11} discuss these issues further.}%
\footnote{An array $\{X_{nt}\}$ is an $L_{\rho}$-\ned\ process on a
  mixing array $\{V_{nt}\}$ if
  \begin{equation}
    \| X_{nt} - \E(X_{nt}
    \mid V_{n,t-m},\dots,V_{n,t+m}) \|_{\rho} \leq d_{nt} v_m
  \end{equation}
  with $v_m \to 0$ as $m \to \infty$ and $\{d_{nt}\}$ an array of
  positive constants.  It is of size $-\gamma$ if $v_m = O(m^{-\gamma
    - \delta})$ for all $\delta>0$.  Dropping the index ``$n$'' gives
  the series definition.  Note that the underlying strong and uniform mixing arrays
  are not required to be stationary.} %
But these conditions are still stronger than
required for a \clt\ to hold; \citet{Jon:97} has established the \clt\
under $L_2$-\ned\ with smaller size and moment
restrictions.%
\footnote{\citet{Jon:97} proves that the \clt\ holds for averages of
  $L_2$-\ned\ processes of size $-1/2$ on a strong mixing series of
  size $-r/(r-2)$, $r > 2$ and the original series having bounded
  $r$-moments.} %
This paper shows that these block bootstrap
methods consistently estimate the distribution of the sample mean
under \citepos{Jon:97} assumptions and show that an \fclt\ holds as
well.%
\footnote{\citet{Rad:96} proves consistency for the Moving Blocks
  Bootstrap for any stationary strong mixing sequence that satisfies
  the \clt. This paper uses a similar method of proof to his, but also
  accommodates nonstationary sequences and the Stationary
  Bootstrap.} %
It also relaxes \citepos{GoW:02} and \citepos{GoJ:03} requirement that
the expected block length be $o(n^{1/2})$ to the original papers'
requirement that it be $o(n)$.

The proof exploits the conditional independence of the blocks in each
bootstrap.  Each bootstrap proceeds by drawing blocks of $M$
consecutive observations from the original time series, and then
pasting these blocks together to create the new bootstrap time series.
The Moving Blocks bootstrap does exactly that; the Circular Block
bootstrap ``wraps'' the observations, so that $(X_{n-1}, X_n, X_1,
X_2)$, for example, is a possible block of length four (letting $X_t$
denote the original time series).  The Stationary Bootstrap wraps the
observations and also draws $M$ at random for each block;
\citet{PoR:94} suggest drawing $M$ from the geometric distribution.
As the name suggests, the series produced by the Stationary Bootstrap
are strictly stationary, while those produced by the other methods are
not.  Although the Stationary Bootstrap was believed to be much less
efficient than other block bootstrap methods due to results of
\citet{Lah:99}, \citet{Nor:09} has shown that it is only slightly less
efficient than the other block-bootstrap methods discussed in this
paper, and has efficiency identical to that of the non-overlapping
block bootstrap.  Consequently, there has been renewed interest in the
Stationary Bootstrap since stationarity of the bootstrap samples can
be a useful property in theoretical research.  \cite{KrP:11} provides
a recent review of the bootstrap for time-series
processes%
\footnote{Also see the discussion papers by \citet{Dah:11},
\citet{GoP:11}, \citet{Hor:11}, \citet{JeM:11}, and
\citet{KrP:11b}.} %
and \citet{GoP:11} further discuss recent developments in
block-bootstraps.

Theorem~\ref{main-bootstrap-clt} presents the main result, asymptotic normality of
the distribution of bootstrapped sums. This paper adopts the
standard notation that $\E^{*}$, $\var^{*}$, etc.\ are the usual
operators with respect to the probability measure induced by the
bootstrap and will use explicit stochastic array notation
for precision.  Also note that all results are presented for the
scalar case but generalize immediately to random vectors.  All of the
proofs are presented in the appendix; only proofs for the Stationary
Bootstrap are presented, since proofs for the other methods are
similar and easier to construct.  All limits are taken as $n \to \infty$
unless otherwise noted and $\lVert \cdot \rVert_r$ denotes the
$L_r$-norm.

\begin{thm}\label{main-bootstrap-clt}
  Suppose the following conditions hold.
  \begin{enumerate}
  \item $X_{nt}$ is $L_2$-\ned\ of size $-1/2$ on an array
    $\{V_{nt}\}$ that is either strong mixing of size $-r/(r-2)$ or
    uniform mixing of size $-r/2(r-1)$, with $r > 2$.  The
    \ned\ magnitude indices are denoted $\{d_{nt}\}$.
  \item The array $\mu_{nt} - \bar \mu_n$ is uniformly bounded
    and $\sum_{t=1}^n (\mu_{nt} - \bar \mu_n)^2 \to 0$,
    where $\E X_{nt} = \mu_{nt}$ and $\bar{\mu}_n = n^{-1} \sum_{t=1}^n
    \mu_{nt}$. Moreover, $\sqrt{n} \| \bar{X}_{n} - \bar\mu_n \|_2
    \to \sigma > 0$, with $\bar X_n = (1/n) \sum_{t=1}^n X_{nt}$.
  \item There exists an array of positive real numbers $\{c_{nt}\}$
    such that $(X_{nt} - \mu_{nt})/c_{nt}$ is uniformly $L_r$-bounded
    and $c_{nt}$ and $d_{nt}/c_{nt}$ are uniformly bounded in $n$
    and~$t$.
  \item $X_{nt}^{*}$ is generated by the Stationary Bootstrap with
    geometric block lengths with success probability $p_n$, $p_n = c
    n^{-a}$ and $a,c \in (0,1)$, or by the Moving or Circular Block
    bootstrap with block length $M_n$ such that $M_n \sim n^a$ for
    $a \in (0,1)$.  Let $M_{ni}$ be the block length of the $i$th
    block, $i=1,\dots,J_n$, and define $K_{n0} = 0$ and $K_{nj} =
    \sum_{i=1}^j M_{ni}$. The last block, $M_{n,J_n}$, is defined as
    $K_{n,J_n} - K_{n,J_n-1}$, so $K_{n,J_n} = n$ a.s.
  \end{enumerate}
  Then $\hat\sigma^{*2} \to^p \sigma^2$,
  \begin{equation}\label{eq:36}
    \sup_x \big\lvert \pr^{*}\big[\sqrt{n}(\bar X_{n}^{*} - \E^* \bar X_n^*) \leq x \big]
    - \pr \big[\sqrt{n}(\bar X_{n} - \E \bar X_n) \leq x \big] \big\rvert \to^p 0,
  \end{equation}
  and
  \begin{equation}
    \label{eq:2}
    \sup_x \big\lvert \pr^{*}\big[
    \sqrt{n}(\bar X_{n}^{*} -  \E^* \bar X_n^*) \big/ \hat\sigma_n^{*}
    \leq x \big] - \Phi(x) \big\rvert \to^p 0,
  \end{equation}
  where $\Phi$ is the \allcaps{cdf} of the Standard Normal
  distribution and
    \begin{equation}
      \label{eq:3}
      \hat{\sigma}_n^{*2} = \tfrac{1}{n} \sum_{j=1}^{J_n}
      \Big\{\sum_{t=K_{n,j-1}+1}^{K_{nj}} (X_{nt}^{*} - \bar X_{n}^{*})\Big\}^2.
    \end{equation}
\end{thm}

As mentioned earlier, these are the same size and mixing conditions
used by \citet{Jon:97}.
Note that De Jong does allow a little bit more flexibility in the
conditions on the array $\{c_{nt}\}$ \citep[see also][]{Dav:93};
essentially, he allows for a single set of blocks with the
maximal $\{c_{nt}\}$ over each block well-behaved, while this
paper requires this condition to hold for every possible partition of
blocks.  This additional restriction is required because the
Stationary Bootstrap will select the blocks randomly and is similar to
\citepos{JoD:00b} requirement for the \fclt. Similarly, our assumption
on the dispersion of the individual means, $(\mu_{nt} - \bar\mu_n)^2$, is
slightly stronger than \citepos{GoW:02} and \citepos{GoJ:03} to
accommodate larger block sizes.

Theorem~\ref{main-bootstrap-clt} relies on a general insight about the
variance of the sample mean under the bootstrap-induced
distribution. It is well-known that a key step in proving the \clt\
for arbitrary dependent processes is demonstrating that the squared
elements converge to a positive and finite limit; i.e.\ if
$\{Z_{nj};~j=1,\dots,J_n\}$ is a representative stochastic array,
$\sum_j Z_{nj}^2 \to^p \sigma^2$ is an important necessary condition
for $\sum_j Z_{nj} \to^d N(0,\sigma^2)$.  (See Section 3.2 of
\citealp{HaH:80}, for further discussion.)  For martingale difference
arrays, each $Z_{nj}$ is one of the original random variables $X_{nt}$
(typically normalized by $1/\sqrt{n}$), but for other forms of
dependence (\ned\ or mixingale arrays, for example, as in
\citealp{Jon:97}) each $Z_{nj}$ is a contiguous block of the original
random variables,
\begin{equation*}
  Z_{nj} = \tfrac{1}{\sqrt{n}} \sum_{t=(j-1) M_n +1}^{j M_n} (X_{nt} - \mu_{nt}),
\end{equation*}
that adds up to the original summation (plus potentially a negligible
residual) so $\sum_j Z_{nj}= (1/\sqrt{n}) \sum_t (X_{nt} - \mu_{nt}) +
o_p(1)$. In \citet{Jon:97}, for example, the \clt\ for mixingale
arrays assumes that there exists a sequence of blocks such that $\sum_j
Z_{nj}^2$ converges i.p., and the \ned\ \clt\ establishes conditions
under which such a $Z_{nj}$ exists.

Our insight is that the expectation of squared blocks of the bootstrap
process can be expressed as a sequence of \emph{contiguous} blocks of
the original process, so the arguments that establish convergence of
the original squared blocks can be applied with only minor changes to
the bootstrapped blocks. Consider the Moving Blocks Bootstrap,%
\footnote{To make this presentation as simple as possible, assume for
  now that $n = M_n J_n$ exactly.} %
for example, and let
\begin{equation*}
  Z_{nj}^* = \tfrac{1}{\sqrt{n}} \sum_{t=(j-1) M_n + 1}^{j M_n} (X_{nt}^* - \bar X_n).
\end{equation*}
Then, conditional on the data, the $Z_{nj}^{*2}$ are independent can
be expected to obey an \lln, so %
$\sum_j (Z_{nj}^{*2} - \E^* Z_{nj}^{*2}) \to^p 0$ %
and the CLT for the bootstrapped array requires $\sum_j \E^*
Z_{nj}^{*2}$ to converge to a positive and finite limit.  But, since
$\E^*$ only averages over the starting point of each block, we have
\begin{align*}
  \E^* Z_{nj}^{*2} &=
  \tfrac{1}{n} \sum_{\tau = 0}^{n-M_n}
  \Big(\tfrac{1}{\sqrt{n}}
  \sum_{t= \tau + 1}^{\tau + M_n} (X_{nt} - \bar X_n) \Big)^2 \\
  &= \tfrac{1}{n} \sum_{\tau_0 = 0}^{M_n-1}
  \sum_{j=1}^{J_n-1}
  \Big(\tfrac{1}{\sqrt{n}}
  \sum_{t= (j - 1) M_n + \tau_0 + 1}^{j M_n + \tau_0} (X_{nt} - \bar X_n) \Big)^2
\end{align*}
after grouping blocks separated by $M_n$ periods. For each $\tau_0$,
the summation can be expected to converge in probability through the
same arguments that were used to establish the \clt\ for the original
array.%
\footnote{Some of the details of the argument will typically need to
  change because the original \clt\ only requires convergence for
  $\tau_0 = 0$, but these details are often incidental to the original
  argument.}%
\footnote{Lemmas~\ref{bootstrap-variance-convergence-lemma} and
  \ref{bootstrapped-zsq-uniform-integrability-lemma} are particularly
  strong demonstrations of this argument.} %
A similar representation is available for the Circular and Stationary
bootstraps.

In short, the basic approach that we use to prove Theorem~\ref{main-bootstrap-clt}
is based on a fundamental connection between the second moments of the
bootstrap process and the sum of squared blocks of the original
array. Even though the details of our proof rely on specific
techniques for \ned\ arrays, this connection implies that block
bootstraps are typically consistent when the original dependent array
obeys the \clt\ and the connection should be useful for proving
consistency of the bootstrap under other dependence conditions.

Theorem~\ref{main-bootstrap-clt} can be further extended to give an
\fclt\ using arguments from \citet{JoD:00b}. We show in
Theorem~\ref{main-bootstrap-invariance-theorem} that the partial sum
of the bootstrapped process obeys an \fclt\ and can be used to derive
critical values for other test statistics under the same assumptions
as Theorem~\ref{main-bootstrap-clt}.

\begin{thm}\label{main-bootstrap-invariance-theorem}
  Suppose that the conditions of Theorem~\ref{main-bootstrap-clt}
  hold, let $W$ be standard Brownian Motion, and define
  \begin{equation}
    \label{eq:6}
    W_n^{*}(\gamma) = \tfrac{1}{\sqrt{n}}
    \sum_{t=1}^{\lfloor \gamma n \rfloor} (X^*_{nt} - \mu_n^{*}) / \sigma_{n}^{*}
  \end{equation}
  and
  \begin{equation}
    \label{eq:7}
    \hat{W}_n^{*}(\gamma) = \tfrac{1}{\sqrt{n}}
    \sum_{t=1}^{\lfloor \gamma n \rfloor} (X^*_{nt} - \mu_n^{*}) / \hat\sigma_n^{*}.
  \end{equation}
  Then $\pr^{*}[\dist(W_n^{*}, W) > \delta] \to 0$ i.p.\ and
  $\pr^{*}[\dist(\hat{W}_n^{*}, W) > \delta] \to 0$ i.p.\ for any
  positive $\delta$ and distance $\dist$ that metricizes weak
  convergence.
\end{thm}

Obviously, Theorem~\ref{main-bootstrap-invariance-theorem}, has
natural corollaries that allow it to be used to approximate the
behavior of partial sums of the original series.  But applying this
result requires stronger assumptions than
Theorem~\ref{main-bootstrap-clt}. The bootstrapped processes are
(essentially) homoskedastic, so they are unable to match patterns of
heteroskedasticity in the original series and we can not estimate the
covariance process of the original partial sum in the bootstrap. This
necessitates a separate consistent estimator of the covariance process, which
we did not need for the sample mean.  Other methods, such as the Local
Block Bootstrap \citep{PaP:02,DPP:03}, may be able to capture this
additional heterogeneity with the bootstrap alone, but we do not
pursue that possibility further.

\begin{cor}
  Suppose the conditions of Theorem~\ref{main-bootstrap-clt} hold and
  that $\sup_{t=1,\dots,n}|\mu_{nt} - \bar \mu_n| = o(1/\sqrt{n})$. Define
  \begin{equation}
    \label{eq:8}
    \sigma_n^2(\gamma) = \tfrac{1}{n}
    \sum_{s,t=1}^{\lfloor \gamma n \rfloor} \cov(X_{ns}, X_{nt}),
  \end{equation}
  let $\hat\sigma_n^2(\cdot)$ be an estimator of $\sigma_n^2(\cdot)$
  such that $\sup_{\gamma} |\sigma_n^2(\gamma) /
  \hat{\sigma}_n^2(\gamma) - 1| \to^p 0$, and let
  \begin{equation}
    \label{eq:9}
    \hat{W}_n(\gamma) = \tfrac{1}{\sqrt{n}} \sum_{t=1}^{\lfloor \gamma
      n \rfloor} (X_{nt} - \bar\mu_n) / \hat{\sigma}_{n}(\gamma).
  \end{equation}
  Then $\pr^{*}[\dist(W_n^{*}, \hat{W}_n) > \delta] \to^p 0$ and
  $\pr^{*}[\dist(\hat{W}_n^{*}, \hat{W}_n) > \delta] \to^p 0$ for any
  positive $\delta$ and any distance $\dist$ that metricizes weak
  convergence.
\end{cor}

The rest of the paper presents the mathematical proofs in detail.

\appendix
\section{Proof of main results}

For both results, we only present a proof for the stationary
bootstrap. The moving blocks and circular block bootstrap follow the
same general argument but are simpler. First we will define some
notation that will be used repeatedly.

Define the \sigmafield
\begin{equation}
  \Ms = \sigma(J_{n}, M_{n1},\dots,M_{nJ_n})
\end{equation}
and the conditional probability $\Pm(\cdot) = \pr[\cdot \mid \Ms]$.
(And define $\Pm^*(\cdot) = \pr^*[\cdot \mid \Ms]$, $\Em(\cdot) =
\E^*(\cdot \mid \Ms)$, etc.) An important property is
that $\Ms$ is \emph{independent} of the $X_{nt}$, so we can treat any
$M_{nj}$ and $J_n$ terms as constants within $\Em(\cdot)$ and
$\Pm[\cdot]$ and integrate over the unconditional distributions of the
$X_{nt}$. This property is especially important because it allows us
to use maximal inequalities and other moment inequalities for
mixingale arrays without modification to construct almost sure bounds on the
conditional moments $\Em$.

Also define
\begin{equation}
  \label{eq:10}
  I_n(\tau, m) = \begin{cases}
    \{\tau + 1,\dots, \tau + m\} &
    \textif\ 0 \leq \tau \leq n - m \text{\ and\ } 1 \leq m \\
    \{1,\dots, m - n + \tau\} \cup \{\tau + 1,\dots, n\} &
    \textif\ n - m < \tau \leq n \text{\ and\ } 1 \leq m\\
    \emptyset & \textif\  m \leq 0,
  \end{cases}
\end{equation}
so each $I_n(\tau, m)$ defines a potential block of length $m$ of the
original observations that could be chosen by the bootstrap.%
\footnote{The index sets $I_n(\tau, m)$ are designed to ``wrap
  around'' and use the first observations when $\tau + m > n$,
  matching the defining aspect of the stationary and circular block
  bootstraps.} %
By convention, summations over empty index sets will be considered
equal to zero. Also let
\newcommand{\Zb}[1][j]{{Z^*_{n{#1}}}}
\begin{align}\label{eq:11}
  Z_n(\tau, m) &= \tfrac{1}{\sqrt{n}} \sum_{t\in I_n(\tau, m)} (X_{nt} - \bar X_n) \\
  Z_n^*(\tau,m)&= \tfrac{1}{\sqrt{n}} \sum_{t \in I_n(\tau,m)} (X_{nt}^{*} - \bar X_n) \label{eq:12}
\intertext{and}
  \Zb          &= Z_n^*(K_{n,j-1}, M_{nj}),
\intertext{and define the corresponding demeaned terms}
  Z_n'(\tau, m) &= \tfrac{1}{\sqrt{n}} \sum_{t\in I_n(\tau, m)} (X_{nt} - \mu_{nt})
  \label{eq:13}\\
  Z_n^{\prime*}(\tau,m)&= \tfrac{1}{\sqrt{n}} \sum_{t \in I_n(\tau,m)} (X_{nt}^{*} - \mu_{nt}^*)
  \label{eq:14}
\intertext{and}
  Z_{nj}^{\prime *} &= Z_n^{\prime*}(K_{n,j-1}, M_{nj}), \label{eq:15}
\end{align}
where $\mu_{nt}^*$ is the expected value of the observation in the
original dataset corresponding to the $t$th observation in the
bootstrapped dataset. Further, define the filtration
\begin{equation}
\Gs_{nj} = \sigma(Z^*_{n1},\dots, Z^*_{nj},~X_{n1},\dots,X_{nn},~\Ms)
\end{equation}
so that $\{Z_{nj}^* / \sigma_n^*, \Gs_{nj}\}$ is a martingale difference
array.

By construction,
\begin{equation}\label{eq:16}
  \tfrac{1}{\sqrt{n}} \sum_{t=1}^n (X_{nt}^{*} - \bar X_n)
  = \sum_{j=1}^{J_n} \Zb
\end{equation}
and
\begin{equation}\label{eq:17}
  \Em^{*}g(\Zb,\dots,\Zb[k]) = \frac{1}{n^{k-j+1}}
  \sum_{\tau_1=0}^{n-1} \cdots \sum_{\tau_{k-j+1}=0}^{n-1}
  g(Z_n(\tau_1, M_{nj}),\dots,Z_n(\tau_{k-j+1}, M_{nk}))
\end{equation}
almost surely for any function $g$ and any $j \leq k$. Equation~\eqref{eq:17}
conditions on the lengths of each block, but averages over their
starting points.

\subsection*{Proof of Theorem~\ref{main-bootstrap-clt}}
{%
\newcommand{\Db}[1][j]{D_{n#1}^{*}}
\newcommand{\Zsum}{\sum_{j=1}^{J_n} \Zb / \sigma_n}
First we prove that
\begin{equation}\label{eq:37}
  \sup_x \big\lvert \pr^{*}\big[
  \sqrt{n}(\bar X_{n}^{*} -  \bar X_n) \big/ \sigma_n^{*}
  \leq x \big] - \Phi(x) \big\rvert \to^p 0
\end{equation}
where $\sigma_n^{*2} = n \E^* (\bar X_n^* - \bar X_n)^2$.
Rewrite $\sqrt{n}(\bar X_{n}^{*} - \bar X_n)$ as in
Equation~\eqref{eq:16}, so
\begin{equation*}
  \tfrac{1}{\sqrt{n}} \sum_{t=1}^n (X_{nt}^* - \bar X_n) =
  \sum_{j=1}^{J_n} Z_{nj}^*
\end{equation*}
and $\{\Zb / \sigma_n^*, \Gs_{nj}\}$ is a martingale difference
array. Moreover,
\begin{equation}
  \label{eq:18}
  \Pm^{*} \Big[\sum_{j=1}^{J_n} \Zb / \sigma_n^* \leq x \Big] - \Phi(x) \to^p 0,
\end{equation}
for all $x$ if $\sigma_n^{*2} \to^p \sigma^2$ (which ensures that
$\sigma_n^{*2}$ is uniformly a.s.\ positive and holds by
Lemma~\ref{bootstrap-variance-convergence-lemma}) and the following
two conditions hold for all positive $\epsilon$:
\begin{equation}
  \label{eq:19}
  \sum_{j=1}^{J_n} \Em^{*} \big(Z_{nj}^{*2} \ind\{Z_{nj}^{*2}  >
  \epsilon\}\big) \to^p 0
\end{equation}
and
\begin{equation}
  \label{eq:20}
  \Pm^{*}\Big[\ \Big|\sum_{j=1}^{J_n} Z_{nj}^{*2} - \sigma_n^{*2}
  \Big|\ > \epsilon \Big] \to^p 0
\end{equation}
since~\eqref{eq:19} and~\eqref{eq:20} ensure that $\Zb/\sigma_n^*$ obeys
a martingale difference \clt\ \citep[e.g.][Theorem 3.3]{HaH:80}.%
\footnote{Conditional on $X_{n1},\dots,X_{nn}$, $J_n$, and
  $M_{n1},\dots,M_{n,J_n}$, the only stochastic components of $\Zsum$
  are the start periods of each block, which are
  $\discreteuniform(1,\dots,n)$ and are independent of all of the
  other random variables in the information set used for conditioning.
  Consequently, $\Pm^*$ is a regular conditional probability and
  arguments like Hall and Heyde's (1980) Theorem 3.3
  apply without modification on this probability measure. See
  also Section 23.2 of \citet{Vaa:00}.
  Also note that Hall and Heyde's Theorem 3.3 as stated imposes an
  additional restriction on the sigma-fields. However, as Hall and
  Heyde discuss on pages 59 and 63--64, that condition is unnecessary
  here because $\sigma_n^{*2}$ is measurable with respect to
  all of the $\Gs_{nj}$.} %

For~\eqref{eq:20}, $Z_{nj}^*$ and $Z_{nk}^*$ (when $k \neq j$) are
conditionally uncorrelated given $X_{n1},\dots,X_{nn},$ and $\Ms$, which
implies
\begin{equation*}
  \sum_{j=1}^{J_n} Z_{nj}^{*2} - \sigma_n^{*2} =
  \sum_{j=1}^{J_n} \Big( Z_{nj}^{*2} - (1/J_n) \E^* \sum_{j=1}^{J_n} Z_{nj}^{*2} \Big)
\end{equation*}
almost surely. But
\begin{equation*}
   \Big\{\tfrac{n}{M_{nj}} \Big(Z_{nj}^{*2} - (1/J_n) \E^* \sum_{j=1}^{J_n} Z_{nj}^{*2}\Big),
   \ \Gs_{nj} \Big\}
\end{equation*}
is a uniformly-integrable martingale difference array
by Lemma~\ref{bootstrapped-zsq-uniform-integrability-lemma}
and satisfies the \lln\ (e.g., \citealp{Dav:94}, Theorem 19.7). So
this sum converges to zero in conditional probability, proving~\eqref{eq:20}.

To prove~\eqref{eq:19}, it suffices to show that
\begin{equation}\label{eq:21}
  \Em \sum_{j=1}^{J_n} \Em^{*}\big(Z_{nj}^{*2} \ind\{Z_{nj}^{*2} > \epsilon\}\big) \to^p 0.
\end{equation}
Lemma~\ref{bootstrapped-zsq-uniform-integrability-lemma} implies that there exists a finite,
monotone function $B(\cdot)$ such that $B(x) \to 0$ as $x \to \infty$
and
\begin{equation*}
  \Em( (n Z_{nj}^{*2} / M_{nj}) \ind\{n Z_n^{*2} / M_{nj} > \epsilon n / M_{nj}\}) \\
  \leq B(\epsilon n /M_{nj})
\end{equation*}
almost surely for large enough $n$ and all $j$ and $\tau$.  Consequently,
\begin{align*}
    \Em \sum_{j=1}^{J_n} \Em^{*}( Z_{nj}^{*2} \ind\{Z_{nj}^{*2} > \epsilon\})
    & \leq \sum_{j=1}^{J_n} ( M_{nj} / n )\, B(\epsilon n /M_{nj}) \\
    & \leq \max_{j = 1,\dots,J_n} B(\epsilon n / M_{nj}) \sum_{i=1}^{J_n} M_{ni} / n \\
    & = B(\epsilon n / \max_{j = 1,\dots,J_n} M_{nj}) \\
    & \to^p 0 \text{\ as\ } n \to \infty,
\end{align*}
where the equality holds by monotonicity of $B$ and convergence in
probability holds by Lemma~\ref{block-length-convergence-lemma},
completing the proof of~\eqref{eq:19}.
The Dominated Convergence Theorem then ensures that
\begin{equation}\label{eq:22}
  \pr^{*} \Big[\sum_{j=1}^{J_n}
  Z_{nj}^{*} / \sigma_n^* \leq x \Big] - \Phi(x) \to^p 0.
\end{equation}
follows from~(\ref{eq:18}). (Also see
Lemma~\ref{equivalent-convergence-from-lie-lemma}.)

Lemma~\ref{bootstrap-variance-convergence-lemma} implies that
$\sigma_n^{*2}$ and $\hat\sigma_n^{*2}$ both converge to $\sigma^2$ in
probability. This convergence then implies that
\begin{equation}
  \label{eq:23}
  \pr^{*}\big[\sqrt{n}(\bar X_{n}^{*} - \bar X_n) \leq x\big] \to^p \Phi(x/\sigma)
\end{equation}
and
\begin{equation}
\label{eq:24}
  \pr^{*}\big[\sqrt{n}(\bar X_{n}^{*} - \bar X_n) / \hat\sigma_n^{*}
  \leq x\big] \to^p \Phi(x)
\end{equation}
for any $x$.  These results are sufficient for~\eqref{eq:36}
and~\eqref{eq:2} though an argument attributed to Poly{\`a}
that proceeds as follows.
Let $k$ be a finite integer and define $x_i = \sigma \Phi^{-1}(i/k)$ for $i =
0,\dots,k$ (so $x_0 = -\infty$ and $x_k = +\infty$).
For any $x \in [x_i, x_{i+1}]$,
\begin{align*}
  \pr^*\big[\sqrt{n} (\bar X_n^* - \bar X_n) \leq x] - \Phi(x/\sigma)
  &\leq \pr^*\big[\sqrt{n} (\bar X_n^* - \bar X_n) \leq x_{i+1}] - \Phi(x_i/\sigma) \\
  &= \pr^*\big[\sqrt{n} (\bar X_n^* - \bar X_n) \leq x_{i+1}] - \Phi(x_{i+1}/\sigma) + 1/k
\intertext{and}
  \pr^*\big[\sqrt{n} (\bar X_n^* - \bar X_n) \leq x] - \Phi(x/\sigma)
  &\geq \pr^*\big[\sqrt{n} (\bar X_n^* - \bar X_n) \leq x_i] - \Phi(x_{i+1}/\sigma) \\
  &= \pr^*\big[\sqrt{n} (\bar X_n^* - \bar X_n) \leq x_i] - \Phi(x_i/\sigma) - 1/k
\end{align*}
almost surely. Then
\begin{multline*}
  \sup_{x \in (-\infty, +\infty)} \big| \pr^*\big[\sqrt{n} (\bar X_n^* - \bar X_n) \leq x] - \Phi(x/\sigma) \big| \\
  \leq \sup_{i=0,\dots,k} \big|\pr^{*}\big[\sqrt{n}(\bar X_{n}^{*} - \bar X_n) \leq x_i \big] - \Phi(x_i/\sigma) \big| + 1/k
\end{multline*}
almost surely and~\eqref{eq:23} ensures that
\begin{equation*}
  \sup_{i=0,\dots,k} \big|\pr^{*}\big[\sqrt{n}(\bar X_{n}^{*} - \bar X_n)
  \leq x_i \big] - \Phi(x_i/\sigma) \big| + 1/k \to^p 1/k
\end{equation*}
for any finite $k$. Since $k$ is arbitrary,~\eqref{eq:37} holds. Since
Theorem~\ref{main-bootstrap-clt}'s assumptions ensure that the original
array obeys the \clt,~\eqref{eq:36} holds \citep[Theorem 2]{Jon:97}.
A similar argument applies to the asymptotic distribution of
$\sqrt{n}(\bar X_n^* - \bar X_n) / \hat\sigma_n^*$, completing the
proof.\qed
}

\subsection*{Proof of Theorem~\ref{main-bootstrap-invariance-theorem}}
\newcommand{\gn}{{\lfloor \gamma n \rfloor}}
\newcommand{\gJ}{{\lfloor \gamma J_n \rfloor}}
\newcommand{\lastK}{{K_{n, \gJ}}}
Define $W_n^{\prime*}$ as
\begin{equation*}
  W^{\prime*}_n(\gamma)
  = \tfrac{1}{\sqrt{n}} \sum_{t=1}^\gn (X_{nt}^* - \bar X_{n})/\sigma
  = \sum_{j=1}^\gJ \Zb /\sigma + Z_n^*(\lastK, \gn - \lastK) / \sigma.
\end{equation*}
Lemma~\ref{bootstrap-variance-convergence-lemma} ensures that
$\sigma_n^{*2}$ and $\hat\sigma_n^{*2}$ both converge in probability to
$\sigma^2$, so it suffices to prove that $\pr^*[\dist(W_n^{\prime*}, W) >
\delta)] \to^p 0$; we can also assume without loss of generality that
$\sigma^2 = 1$. Moreover,
Lemma~\ref{equivalent-convergence-from-lie-lemma} ensures that it
suffices to prove unconditional convergence, so we will establish
$\pr[\dist(W_n^{\prime *}, W) > \delta] \to 0$.

Theorem~\ref{main-bootstrap-clt} implies that, for any fixed $\gamma$,
$W_n^{\prime *}(\gamma)$ is asymptotically normal and, as a
partial sum of an \mds, $W_n^{\prime *}$ has asymptotically independent
increments. To see this, choose $\gamma, \gamma' \in [0,1]$ and
$\delta, \delta' > 0$ so that $\delta + \gamma \leq \gamma'$. Since
the increments of $W_n^{\prime *}$ are uncorrelated conditional on
$X_{1n},\dots,X_{nn}$, and $\Ms$, we have
\begin{equation*}
    \E\big| (W_n^{\prime *}(\delta' + \gamma') - W_n^{\prime *}(\gamma'))
  (W_n^{\prime *}(\delta + \gamma) - W_n^{\prime *}(\gamma)) \big| = 0
\end{equation*}
for large enough $n$ if $\gamma' > \gamma + \delta$. If $\gamma' =
\gamma + \delta$ then
\begin{multline*}
  \E\big[(W_n^{\prime *}(\delta' + \gamma') - W_n^{\prime *}(\gamma'))
  (W_n^{\prime *}(\delta + \gamma) - W_n^{\prime *}(\gamma)) \big] = \\
  \E \Em^*\big\{- Z_n^*(\lfloor \gamma' n \rfloor, K_{n, \lceil \gamma' J_n \rceil }- \lfloor \gamma' n \rfloor)
  Z_n^*(K_{n,\lfloor (\gamma + \delta) J_n \rfloor},
  \lfloor (\gamma + \delta) n \rfloor - K_{n,\lfloor (\gamma + \delta) J_n \rfloor}))\big\}.
\end{multline*}
But this second quantity can be bounded:
\begin{align*}
  \E \Em^*\big\{-& Z_n^*(\lfloor \gamma' n \rfloor,
  K_{n, \lceil \gamma' J_n \rceil }- \lfloor \gamma' n \rfloor)
  Z_n^*(K_{n,\lfloor (\gamma + \delta) J_n \rfloor},
  \lfloor (\gamma + \delta) n \rfloor - K_{n,\lfloor (\gamma + \delta) J_n \rfloor}))\big\} \\
  &\leq \big\{\E \Em\big[Z_n^*(\lfloor \gamma' n \rfloor, K_{n, \lceil \gamma' J_n \rceil }- \lfloor \gamma' n \rfloor)\big]^2 \\
  & \quad \times \big\{\E \Em Z_n^*(K_{n,\lfloor (\gamma + \delta) J_n \rfloor},
  \lfloor (\gamma + \delta) n \rfloor - K_{n,\lfloor (\gamma + \delta) J_n \rfloor}))^2\big\}^{1/2} \\
  &\leq C \E M_{n, \lfloor (\gamma + \delta) J_n \rfloor} / n
\end{align*}
for some constant $C$ by
Lemma~\ref{bootstrapped-zsq-uniform-integrability-lemma}. This term
converges to zero by Lemma~\ref{block-length-convergence-lemma}.

Consequently, as in \cite{JoD:00b}, the result follows from stochastic
equicontinuity \citep[Theorems 15.4 and 15.5]{Bil:68} namely that
\begin{equation*}
  \lim_{\delta \to 0} \limsup_{n \to \infty} \pr[\sup_{\gamma \in [0,1]}
  \sup_{\gamma' \in [\gamma - \delta, \gamma + \delta]}
    | W_n^{\prime *}(\gamma) - W_n^{\prime *}(\gamma') | > \epsilon] = 0
\end{equation*}
for any positive $\epsilon$.  Fix $\delta> 0$ such that $D = 2/\delta$
is a positive integer and let $\gamma_d = d/D$ for $d =
0,\dots,D$. Then, mimicking the argument in \cite{JoD:00b},
\begin{align*}
  \pr[\sup_{\gamma \in [0,1]} &
  \sup_{\gamma' \in [\gamma - \delta, \gamma + \delta]}
  | W_n^{\prime *}(\gamma) - W_n^{\prime *}(\gamma') | > \epsilon] \\
  &\leq \pr[ \sup_{d = 1,\dots,D} \sup_{\gamma \in [0,\delta]}
  | W_n^{\prime *}(\gamma + \gamma_d) - W_n^{\prime *}(\gamma_d) | > \epsilon/2] \\
  &\leq  (4/\epsilon^2) \sum_{d=1}^D \E[\sup_{\gamma \in [0,\delta]}
  | W_n^{\prime *}(\gamma + \gamma_d) - W_n^{\prime *}(\gamma_d) |^2 \\
  &\qquad\qquad\qquad \times \ind\{\sup_{\gamma \in [0,\delta]}
  | W_n^{\prime *}(\gamma + \gamma_d) - W_n^{\prime *}(\gamma_d) |^2 > \epsilon^2/4\}] \\
  &\leq  (4/\epsilon^2) \max_{d=1,\dots,D} \E[\sup_{\gamma \in [0,\delta]}
  (1/\delta) | W_n^{\prime *}(\gamma + \gamma_d) - W_n^{\prime *}(\gamma_d) |^2 \\
  &\qquad\qquad\qquad \times \ind\{\sup_{\gamma \in [0,\delta]}
  (1/\delta) | W_n^{\prime *}(\gamma + \gamma_d) - W_n^{\prime *}(\gamma_d) |^2 > \epsilon^2/4\delta\}].
\end{align*}
Lemma~\ref{bootstrapped-zsq-uniform-integrability-lemma} implies that
\begin{equation*}
  \sup_{\gamma \in [0,\delta]}
  (1/\delta) | W_n^{\prime *}(\gamma + \gamma_d) - W_n^{\prime *}(\gamma_d) |^2
\end{equation*}
is uniformly integrable, so there exists a finite and positive
function $B$ such that $B(x) \to 0$ as $x \to \infty$ and
\begin{multline*}
  \E[\sup_{\gamma \in [0,\delta]}
  (1/\delta) | W_n^{\prime *}(\gamma + \gamma_d) - W_n^{\prime *}(\gamma_d) |^2
  \ind\{\sup_{\gamma \in [0,\delta]}
  (1/\delta) | W_n^{\prime *}(\gamma + \gamma_d) - W_n^{\prime *}(\gamma_d) |^2 > x\}] \\
  \leq B(x)
\end{multline*}
for all $d$ and $\delta$ and all large enough $n$.

As a result,
\begin{align*}
  \lim_{\delta \to 0} \limsup_{n\to\infty} \pr[\sup_{\gamma \in [0,1]}
  \sup_{\gamma' \in [\gamma - \delta, \gamma + \delta]}
  | W_n^{\prime *}(\gamma) - W_n^{\prime *}(\gamma') | > \epsilon] & \leq
  \lim_{\delta \to 0} (4/\epsilon^2)
  B(\epsilon^2/4\delta) \\
  &= 0,
\end{align*}
completing the proof.\qed

\section{Supplemental results}

\begin{lem}\label{block-length-convergence-lemma}
  Suppose that $M_{n1}, M_{n2},\dots$ are i.i.d.\ geometric random
  variables with success parameter $p_n = c n^{-a}$ with $a, c \in
  (0,1)$, and that $\ell_n = (p_n \log p_n^{-1})^{-1}$ and define
  $J_n$ so that $\sum_{i=1}^{J_n-1} M_{ni} < n \leq \sum_{i=1}^{J_n} M_{ni} $
  Then
  \begin{enumerate}
  \item $\max_{i=1,\dots,\lfloor C n p_n \rfloor} M_{ni} / n \to^p 0$
    for any positive $C$,
  \item $\max_{i=1,\dots,J_n} M_{ni} / n \to^p 0$,
  \item $\max_{i=1,\dots,  \lfloor C n p_n \rfloor} M_{ni} /
  \ell_n^{1+\epsilon} \to^p 0$ as $n \to \infty$ for any positive
  $\epsilon$ and $C$,
  \item $\max_{i=1,\dots,J_n} M_{ni} /
  \ell_n^{1+\epsilon} \to^p 0$ as $n \to \infty$ for any positive
  $\epsilon$, and
  \item $\sum_{i=1}^{J_n} M_{ni}^2 / n^2 \to^p 0$.
  \end{enumerate}
\end{lem}

\begin{proof}[Proof of Lemma~\ref{block-length-convergence-lemma}]
  To prove part 1, observe that, for any increasing positive sequence
  $x_n$ such that $x_n p_n \to \infty$,
  \begin{equation*}
    \pr\Big[\max_{i=1,\dots, \lfloor C n p_n \rfloor} M_{ni} \leq x_n\Big] =
    (1 - (1 - p_n)^{x_n})^{\lfloor C n p_n \rfloor} \to \lim \exp(-C n p_n (1 - p_n)^{x_n})
  \end{equation*}
  and $C n p_n (1 - p_n)^{x_n} \to \lim C n p_n e^{-x_n p_n}$. Now, let
  $x_n = n x$ for any positive number $x$.  Then
  \begin{equation*}
    \pr\Big[\max_{i=1,\dots,\lfloor C n p_n \rfloor} M_{ni}/n \leq x \Big] \to
    \lim \exp(-C n p_n e^{-n p_n x}) =
    \exp(0) = 1.
  \end{equation*}
  Since $x$ is arbitrary, $\max_{i=1,\dots,\lfloor C n p_n \rfloor} M_{ni} / n \to^p 0$.

  For part 2, take $C$ to be an arbitrary constant strictly greater
  than one. For any $x$,
  \begin{align*}
    \pr\Big[\max_{i=1,\dots,J_n} M_{ni} > x\Big] & \leq
    \pr\Big[\max_{i=1,\dots,\lfloor C n p_n \rfloor} M_{ni} > x
    \text{\ or\ } J_n > \lfloor C n p_n \rfloor\Big] \\
    & \leq \pr\Big[\max_{i=1,\dots,\lfloor C n p_n \rfloor} M_{ni} > x\Big]
    + \pr\Big[\sum\nolimits_{i=1}^{\lfloor C n
      p_n \rfloor} M_{ni} < n \Big]
  \end{align*}
  The first term converges to zero by part 1 and the second term by
  the \lln.

  For part 3, let $x_n = \ell_n^{1 + \epsilon} x$ and note that
  \begin{equation*}
    p_n \ell_n^{1+\epsilon} \geq p_n^{-(\epsilon-\delta-\epsilon\delta)} =
    c^{-(\epsilon-\delta-\epsilon\delta)}
    n^{a(\epsilon-\delta-\epsilon\delta)} \equiv b n^{a(\epsilon
      - \delta - \epsilon\delta)}
  \end{equation*}
  for any $\delta > 0$ and large enough $n$.  Choose $\delta$ small
  enough that $\epsilon > \delta(1 +\epsilon)$. Then
  \begin{equation*}
    n p_n \exp(-\ell_n^{1+\epsilon} p_n) \leq n p_n
    \exp(-b n^{a(\epsilon -
      \delta - \epsilon\delta)}) = c
    v_n^{\frac{1-a}{a(\epsilon-\delta-\epsilon\delta)}}
    \exp(-b v_n) \to 0,
  \end{equation*}
  with $v_n = n^{a(\epsilon-\delta-\epsilon\delta)}$.  Consequently,
  \begin{equation*}
    \pr[\max_i M_{ni}/\ell_n^{1+\epsilon} \leq x ] \to \exp(0) = 1
  \end{equation*}
  as well.

  The proof of part 4 is the same as part 2, making the obvious
  substitutions. Part 5 holds because $\sum_{i=1}^{J_n} M_{ni}^2/n^2
  \leq \max_{i=1,\dots,J_n} M_{ni}/n$ which converges to zero in
  probability by part 2.
\end{proof}

\begin{lem}\label{equivalent-convergence-from-lie-lemma}
  If $\{A_n\}$ is a sequence of events then the following are equivalent:
  \begin{align*}
  \pr[A_n] &\to 0, &
  \pr^*[A_n] &\to 0 \text{\ in\ }L_1, &
  &\text{and}&
  \Pm^*[A_n] &\to 0 \text{\ in\ }L_1.
  \end{align*}
\end{lem}
\begin{proof}
  Since $|\pr[A_n]| = \E |\pr^*[A_n]| = \E |\pr^*_{\mathcal{M}}[A_n]|$ these
  conditions are equivalent by definition.
\end{proof}

\begin{lem}\label{bootstrap-variance-convergence-lemma}
Under the conditions of Theorem~\ref{main-bootstrap-clt},
\begin{gather}
  \label{eq:25}
  \sum_{j=1}^{J_n} (Z_{nj}^{*2} - \Em^* Z_{nj}^{*2}) \to^p 0,
  \\
  \label{eq:26}
  \sum_{j=1}^{J_n} (Z_{nj}^{*2} - Z_{nj}^{\prime*2}) \to^p 0,
\end{gather}
and $\pr^*[ |\sigma_n^{*2} - \sigma^2| > \epsilon] \to^p 0$.
If, in addition, $\bar X_n^* - \bar X_n = O_p(1/\sqrt{n})$ then
$\pr^*[ |\hat\sigma_n^{*2} - \sigma^2| > \epsilon] \to^p 0$.
\end{lem}

\begin{proof}
  For~\eqref{eq:26}, $(Z_{nj}^{*2} - \Em^* Z_{nj}^{*2}) \cdot (n /
  M_{nj})$ is a uniformly integrable martingale difference array, by
  Lemma~\ref{bootstrapped-zsq-uniform-integrability-lemma}, and
  satisfies the \lln. (See~\citealp{Dav:94}, Theorem 19.7.)
  For~\eqref{eq:26}, observe that
  \begin{equation*}
    \Big| \sum_{j=1}^{J_n} (Z_{nj}^{*2} - Z_{nj}^{\prime*2}) \Big| \leq
    2 \Big(\sum_{j=1}^{J_n} Z_{nj}^{*2} \Big)^{1/2}
    \Big(\tfrac{1}{n} \sum_{t=1}^n (\mu_{nt} - \bar X_n)^2\Big)^{1/2}
    + \tfrac{1}{n} \sum_{t=1}^n (\mu_{nt} - \bar X_n)^2
  \end{equation*}
  after several applications of the Cauchy-Schwarz
  inequality. Lemma~\ref{lln-for-real-zsq}, along with~\eqref{eq:17}
  and~\eqref{eq:25}, implies that $\sum_j Z_{nj}^{*2} = O_p(1)$;
  $(1/n) \sum_t (\mu_{nt} - \bar X)^2$ converges to zero in
  probability by assumption on $\mu_{nt} - \bar\mu_n$ and because
  $\bar X$ itself obeys the \lln.

  To show that $\sigma_n^{*2}$ converges, we can write
  \begin{multline*}
    \sigma_n^{*2} - \sigma^2 = \E^*\Big\{ \sum_{j=1}^{J_n} (Z_{nj}^{*2} - Z_{nj}^{\prime*2}) \Big\}
    + \tfrac{1}{n} \sum_{\tau=0}^{n-1} \E^* \Big\{\sum_{j=1}^{J_n}
    \big( Z_{n}^{\prime}(\tau,M_{nj})^2 - \Em Z_{n}^{\prime}(\tau,M_{nj})^2 \big)\Big\} \\
    + \tfrac{1}{n} \sum_{\tau=0}^{n-1} \E^* \Big\{\sum_{j=1}^{J_n}
    \Em Z_{n}^{\prime}(\tau,M_{nj})^2  - \sigma^2\Big\}.
  \end{multline*}
  Uniform integrability ensures that the convergence in~(\ref{eq:25})
  holds in $L_1$ as well and
  Lemma~\ref{equivalent-convergence-from-lie-lemma} then implies that
  the first term converges to zero in probability.
  Lemma~\ref{lln-for-real-zsq} proves that the second and third
  summation converge to zero in probability.

  Next,
  \begin{equation*}
    \hat \sigma_n^{*2} - \sigma_n^{*2}
    = \sum_{j=1}^{J_n} \big( Z_{nj}^* + (M_{nj}/\sqrt{n}) (\bar X_n - \bar X_{n}^*) \big)
    - \E^*  \sum_{j=1}^{J_n} Z_{nj}^{*2}
  \end{equation*}
  so, in light of the previous arguments, $\hat\sigma_n^{*2} \to^p \sigma^2$ if
  \begin{equation}
    (\bar X_n - \bar X_n^*)^2 \sum_{j=1}^{J_n} M_{nj}^2/n \to^p 0,
  \end{equation}
  which holds by Lemma~\ref{block-length-convergence-lemma} and
  assumption.
\end{proof}

\begin{lem}\label{lln-for-real-zsq}
  If the conditions of Theorem~\ref{main-bootstrap-clt} hold then
  \begin{gather}
    \pr\Big[ \Big\lvert
    \tfrac{1}{n} \sum_{\tau=0}^{n-1} \sum_{j=1}^{J_n}
    \big[ Z_n'(\tau, M_{nj})^2 - \Em Z_n'(\tau, M_{nj})^2 \big]
    \Big\rvert > \epsilon \Big] \to 0 \label{eq:27}
    \intertext{and}
    \pr\Big[ \Big\lvert
    \tfrac{1}{n} \sum_{\tau=0}^{n-1} \sum_{j=1}^{J_n}
    \Em Z_n'(\tau, M_{nj})^2 - \sigma^2
    \Big\rvert > \epsilon \Big] \to 0.\label{eq:28}
  \end{gather}
\end{lem}
\newcommand{\nMj}{\lfloor n / M_{nj} \rfloor}

\noindent For these two proofs, let $\ell_n = (p_n
\log p_n^{-1})^{-1}$ and let $L_{nj} = \nMj$; $\ell_n$ represents a
smaller block size that satisfies $\ell_n J_n / n \to^p 0$.

\begin{proof}[Proof of~(\ref{eq:27})]
  We can express this summation as
  \begin{multline}\label{eq:29}
    \tfrac{1}{n} \sum_{\tau=0}^{n-1}  \sum_{j=1}^{J_n}
    \big\{Z_n'(\tau, M_{nj})^2 - \Em Z_n'(\tau, M_{nj})^2 \big\}
    \\ =
    \tfrac{1}{n} \sum_{j=1}^{J_n} \sum_{\tau=0}^{M_{nj}-1}
    \sum_{i=0}^{L_{nj}-1}\big\{\big[ Z_n'(\tau + i M_{nj}, M_{nj} - \ell_n) + Z_n'(\tau + (i+1) M_{nj} - \ell_n, \ell_n)\big]^2 \\
    - \Em\big[ Z_n'(\tau + i M_{nj}, M_{nj} - \ell_n) + Z_n'(\tau + (i+1) M_{nj} - \ell_n, \ell_n)\big]^2 \big\} \\
    + \tfrac{1}{n} \sum_{j=1}^{J_n} \sum_{\tau= M_{nj}L_{nj}}^{n-1}
    \big\{Z_{n}'(\tau, M_{nj})^2 - \Em Z_n'(\tau, M_{nj})^2 \big\}
  \end{multline}
  almost surely.
  By Lemma~\ref{dejong-restatement-lemma} (Equation~\ref{eq:32}), for
  any $\delta > 0$ there exist positive $C$ and $\epsilon$ such that
  \begin{align*}
    \Big\lVert\tfrac{1}{n} \sum_{j=1}^{J_n} & \sum_{\tau=0}^{M_{nj}-1}
    \sum_{i=0}^{L_{nj}-1} \big\{ Z_n'(\tau + i M_{nj}, M_{nj} - \ell_n)^2 -
    \Em \big(Z_n'(\tau + i M_{nj}, M_{nj} - \ell_n)^2\big) \big\}\Big\rVert_1 \\
    &\leq \E \tfrac{1}{n} \sum_{j=1}^{J_n} \sum_{\tau=0}^{M_{nj}-1}
    \Em \Big| \sum_{i=0}^{L_{nj}-1} \big\{ Z_n'(\tau + i M_{nj}, M_{nj} - \ell_n)^2 -
    \Em \big(Z_n'(\tau + i M_{nj}, M_{nj} - \ell_n)^2\big) \big\} \Big|\\
    &\leq \E \Big(\tfrac{1}{n} \sum_{j=1}^{J_n} \sum_{\tau=0}^{M_{nj}-1} 
    \big(2 \delta + C \, \tfrac{M_{nj}}{n^{1/2} \ell^{1/2+\epsilon}}\big)\Big)
  \end{align*}
  for large enough $n$, which converges to $2 \delta$ by
  Lemma~\ref{block-length-convergence-lemma}. Also,
  Lemma~\ref{zsq-uniform-integrability-lemma} ensures that there
  exists a value $C$ (possibly different from the value above) such
  that
  \begin{equation*}
    \E\Big(\tfrac{1}{n}  \sum_{j=1}^{J_n} \sum_{\tau=0}^{M_{nj}-1} \sum_{i=0}^{L_{nj}-1}
    Z_n'(\tau + (i+1) M_{nj} - \ell_n, \ell_n)^2 \Big)
    \leq C \E \Big(\sum_{j=1}^{J_n} L_{nj} \ell_n M_{nj} / n^2\Big)
  \end{equation*}
  and
  \begin{equation*}
    \E \Big(\tfrac{1}{n} \sum_{j=1}^{J_n} \sum_{\tau= M_{nj}L_{nj}}^{n-1} Z_{n}'(\tau, M_{nj})^2\Big)
    \leq
    C \E \Big(\sum_{j=1}^{J_n} M_{nj}^2 / n^2 \Big)
  \end{equation*}
  for large enough $n$, both of which converge to zero in $L_1$ as $n
  \to \infty$ by Lemma~\ref{block-length-convergence-lemma}.  These
  three convergence results imply that the \textsc{rhs} of
  (\ref{eq:29}) converges to zero, completing the proof.
\end{proof}

\begin{proof}[Proof of~(\ref{eq:28})]
  After using similar arguments to in the previous part of the proof,
  the conclusion holds if
  \begin{equation*}
    \tfrac{1}{n} \sum_{j=1}^{J_n} \sum_{\tau=0}^{M_{nj}-1} \sum_{i=0}^{L_{nj} - 1}
    \Em Z_n'(\tau + i M_{nj}, M_{nj} - \ell_n)^2 \to^p \sigma^2,
  \end{equation*}
  which is a direct implication of Lemma~\ref{dejong-restatement-lemma}.
\end{proof}

\begin{lem}\label{bootstrapped-zsq-uniform-integrability-lemma}
  Under the conditions of Theorem~\ref{main-bootstrap-clt},
  \begin{multline}\label{eq:35}
    \lim_{C \to 0} \limsup_{n \to \infty} \sup_{m' = 1,\dots,n}
    \E\Big(\Big(\sum_{\tau=0}^{n-1} \max_{m=1,\dots,m'} Z_n(\tau, m)^2 / m'\Big) \\
    \times\ind\Big\{\sum_{\tau=0}^{n-1} \max_{m=1,\dots,m'} Z_n(\tau, m)^2 / m' > C\Big\}\Big) = 0.
  \end{multline}
  and
  \begin{multline}\label{eq:30}
    \lim_{C \to \infty} \limsup_{n \to \infty} \sup_{\substack{\tau = 0,\dots,n-1 \\ m'=1,\dots,n}}
    \E\Big(\max_{m = 1,\dots,m'} (n \, Z^*_n(\tau, m)^2 / m') \\
    \times \ind\Big\{\max_{m = 1,\dots,m'} n \, Z^*_n(\tau, m)^2 / m' > C\Big\}\Big)
    = 0,
  \end{multline}
  making the families of random variables $\{\max_{m=1,\dots,m'}
  Z_n(\tau, m)^2 n / m'\}$ and $\{\max_{m=1,\dots,m'}
  Z^{*}_n(\tau, m)^2 n / m'\}$ uniformly integrable.
\end{lem}

\begin{proof}
  For \eqref{eq:35}, take $\epsilon > 0$. The sum
  \[
  \sum_{\tau=0}^{n-1} \max_{m=1,\dots,m'} Z_n(\tau, m)^2 / m'
  \]
  is uniformly integrable if and only if it is uniformly $L_1$-bounded
  and there exists an $\epsilon'$ such that
  \[
  \sup_{n, m} \E\Big( \ind(A) \sum_{\tau=0}^{n-1} \max_{m=1,\dots,m'} Z_n(\tau, m)^2 / m' \Big)
  < \epsilon
  \]
  for every event $A$ satisfying $\pr(A) < \epsilon'$.
  (\citealp{Dav:94}, 12.9.)
  Lemma~\ref{zsq-uniform-integrability-lemma} ensures that there is an
  $\epsilon'$ such that
  \begin{equation*}
    \sup_{n, m', \tau} \E\big( \ind(A) \max_{m=1,\dots,m'} Z_n'(\tau, m)^2 n / m' \big) \leq \epsilon / 9
  \end{equation*}
  for all $A$ with $\pr(A) \leq \epsilon'$. Then, writing
  \begin{align*}
    Z_n(\tau, m) &= Z_n'(\tau, m)
    + \tfrac{1}{\sqrt{n}} \sum_{t \in I_n(\tau, m)}
    \big((\mu_{nt} - \bar \mu_n) + (\bar \mu_n - \bar X_n)\big) \\
    &= Z_n'(\tau, m)
    + \tfrac{1}{\sqrt{n}} \sum_{t \in I_n(\tau, m)} (\mu_{nt} - \bar \mu_n)
    - m Z'_n(0, n) \\
  \end{align*}
  gives the relationship
  \begin{align*}
    \E\Big(\ind(A) \sum_{\tau=0}^{n-1} & \max_{m=1,\dots,m'} Z_n(\tau, m)^2/m'\Big) \\
    &\leq 3 \sum_{\tau=0}^{n-1} \E\big( \ind(A)  \max_{m=1,\dots,m'} Z'_n(\tau, m)^2/m'\big) \\
    &\quad+(3/n m') \E\Big(\ind(A) \sum_{\tau=0}^{n-1} \max_{m=1,\dots,m'}
    \Big(\sum_{t \in I_n(\tau, m)} (\mu_{nt} - \bar \mu_n) \Big)^2\Big) \\
    &\quad+ (3/m') \sum_{\tau=0}^{n-1}
    \E (\ind(A) \max_{m=1,\dots,m'} m Z'_n(0, n)^2) \\
    &\leq (2 \epsilon/3) + \pr(A) (3/n)\sum_{t=1}^n (\mu_{nt} - \bar \mu_n)^2
  \end{align*}
  where the last inequality holds as a consequence of $A$'s
  construction. Since $(1/n)\sum_{t=1}^n (\mu_{nt} - \bar \mu_n)^2 \to
  0$, choose $n$ large enough that this average is less than
  $\epsilon/9$. The proof that that the first moment is bounded is
  similar, which completes the proof of~(\ref{eq:35})

  For~(\ref{eq:30}), we will first show that, for every $\epsilon > 0$,
  there exists an $\epsilon' > 0$ with the property that
  \begin{equation}\label{eq:31}
    \Em \Big(\ind(A) \times \max_{m = 1,\dots,m'} Z^*_n(\tau, m)^2 n / m'\Big) < \epsilon
  \end{equation}
  for any event $A \in \Gs \equiv \sigma(X_{11},\dots,X_{nn};\
  M_{n1},\dots,M_{n,J_n};\ J_n)$ with $\pr(A) \leq \epsilon'$. Then we
  will show that this property implies uniform integrability.%
\footnote{The difference between this proof and the proof
  of~\eqref{eq:35} is the measurability requirement. Here we need to
  finesse measurability of the sets $A$.} %

  Take an arbitrary $\epsilon > 0$ and a value of $\epsilon' > 0$
  so that
  \begin{equation*}
    \Em \Big(\ind(A) \sum_{\tau = 0}^{n-1} \tfrac{1}{m'} \max_{m = 1,\dots,m'}
    Z_n(\tau, m)^2\Big) < \epsilon / 4
  \end{equation*}
  for any $A \in \Gs$ with $\pr(A) \leq \epsilon'$, which is shown to
  exist in the first part of this Lemma. Then define $J(x)$ to be the
  block index such that $K_{n,J(x) - 1} < x \leq K_{n,J(x)}$. We have
  the relationships
  \begin{align*}
    \Em \Em^* \Big(&\ind(A) \times \max_{m = 1,\dots,m'} n \, Z^*_n(\tau, m)^2 / m'\Big) \\
    &= \Em \ind(A) \Em^* \Big[ \tfrac{n}{m'}\max_{m = 1,\dots, m'}
    \Big(\sum_{j=J(\tau)+1}^{J(\tau + m)-1} Z_{nj}^*
    + Z_n^*(\tau, K_{n,J(\tau)} - \tau) \\
    &\quad+ Z_n^*(K_{n,J(\tau+m)-1}, m - K_{n,J(\tau + m)-1}) \Big)^2\Big] \\
    &\leq 4 \Em \ind(A) \tfrac{1}{m'} \sum_{u = 0}^{n-1} \Big( \sum_{j=J(\tau)+1}^{J(\tau + m')-1}
    Z_n(u, M_{nj})^2 + Z_n(u, K_{n,J(\tau)} - \tau)^2 \\
    &\quad + Z_n(u, m' - K_{n,J(\tau+m')-1})^2\Big) \\
    &= (4/m') \sum_{j=J(\tau)+1}^{J(\tau + m')-1} \Em\Big[ \ind(A) \sum_{u = 0}^{n-1} Z_n(u, M_{nj})^2\Big] \\
    &\quad + (4/m') \Em\Big[\ind(A) \sum_{u = 0}^{n-1} Z_n(\tau_{J(\tau)}, M_{n,J(\tau)})^2\Big] \\
    &\quad + (4/m') \Em\Big[ \ind(A)  \sum_{u = 0}^{n-1} Z_n(u, m' - K_{n,J(\tau+m')})^2\Big] \\
    &\leq \epsilon.
  \end{align*}
  The first inequality is a consequence of Doob's maximal inequality
  for martingales \citep[see][Theorem 15.15, for example]{Dav:94},
  since $Z_{nj}^*$ is a martingale difference array, and the second
  equality and last inequality follow by construction of $A$. A
  similar argument also implies that $n Z_n^*(\tau, m)^2 / m'$ has
  finite first moment.

  Now, for uniform integrability, take $\epsilon > 0$, choose
  $\epsilon'$ so that~\eqref{eq:31} holds for all $A \in \Gs$
  s.t. $\pr(A) \leq \epsilon'$, and choose $C_0$ large enough that
  \begin{equation*}
    \Pm\Big[ \max_{m = 1,\dots,m'} n \, Z^*_n(\tau, m)^2 / m' > C_0 \Big] \leq \epsilon'.
  \end{equation*}
  Markov's inequality guarantees the existence of this $C_0$. Then
  \begin{equation*}
    \Em \Big(\max_{m = 1,\dots,m'} n \, Z^*_n(\tau, m)^2 / m'
    \times \ind\Big\{ \max_{m = 1,\dots,m'} n \, Z^*_n(\tau, m)^2 / m' > C \Big\}\Big) \leq \epsilon
  \end{equation*}
  for all $C > C_0$. Since $\epsilon$ is arbitrary, this completes the proof.
\end{proof}

\begin{lem}\label{dejong-restatement-lemma}
\newcommand{\isum}{\sum_{i=0}^{\lfloor n/m \rfloor - 1}}
  Suppose the conditions of Theorem~\ref{main-bootstrap-clt} hold.
  For any positive $\delta$, there exist positive and finite constants
  $C$, $n_0$, and $\epsilon$ such that
  for all $n > n_0$, $m = 1,\dots,n$, $\tau = 0,\dots,m$, and $\ell =
  1,\dots,m-1$:
\begin{multline}\label{eq:32}
    \E \Big\lvert \isum \big[Z_n'(\tau + i m, m - \ell)^2
    - \E \big(Z_n'(\tau + i m, m - \ell)^2\big) \big]\Big\rvert \\
    \leq 2 \delta + C \cdot \big(\tfrac{m}{n}\big)^{1/2}
    \big(\tfrac{m}{\ell^{1+\epsilon}}\big)^{1/2}.
\end{multline}
Also, there exists a constant $C$ and a finite function $D(x)$ such
that $D(x) \to 0$ as $x \to \infty$ and, for large enough $n$,
  \begin{equation}
    \label{eq:33}
    \E \Big\lvert \isum \E(Z_n'(\tau + i m, m - \ell)^2
    - \sigma^2 \Big\rvert \leq C\, D(\ell).
  \end{equation}
\end{lem}

Results~\eqref{eq:32} and~\eqref{eq:33} are direct extensions of
\citepos{Jon:97} Lemmas 5 and 4, respectively, replacing De Jong's
implicit use of inequalities with explicit inequalities. The
supplemental appendix presents a proof of~\eqref{eq:33} to show the
main idea.

\begin{lem}\label{zsq-uniform-integrability-lemma}
\newcommand{\uiterm}{\max_{m \in 1,\dots,m'} \Big(
  \sum_{t \in I_n(\tau, m)} (X_{nt} - \mu_{nt})\Big)^2 \Big/
  \sum_{t\in I_n(\tau, m')} c_{nt}^2}
\newcommand{\uitermb}{(1/m') \max_{m \in 1,\dots,m'} \Big(
  \sum_{t \in I_n^*(\tau, m)} (X_{nt} - \mu_{nt})\Big)^2}
  Under the conditions of Theorem~1,
  \begin{multline}\label{eq:34}
    \lim_{C \to 0} \limsup_{n \to \infty} \sup_{\substack{\tau = 0,\dots,n-1 \\  m' = 1,\dots,n}}
    \E\big(\big(\max_{m=1,\dots,m'} Z'_n(\tau, m)^2 n / m'\big) \\
    \times\ind\{\max_{m=1,\dots,m'} Z'_n(\tau, m)^2 n / m' > C\}\big)) = 0.
  \end{multline}
\end{lem}

\begin{proof}
  See supplemental appendix for the proof of~\eqref{eq:34}, which
  follows \citet[Lemma 6.5]{Mcl:75b} and \citet[Lemma 3.5]{Mcl:77}
  almost exactly and is also presented as Theorem~16.13 in
  \citet{Dav:94}.
\end{proof}

\bibliography{texextra/references}
\end{document}

% LocalWords:  CLT Kun LiS PoR GoW GoJ nt indices eq studentized JoD reindex nj
% LocalWords:  De Jong's ns Mcl AllRefs nn th nm ni HaH formulae jn gcalhoun nN
% LocalWords:  jel PaP DPP np Ames Resampling nonstationary cdf Helle
