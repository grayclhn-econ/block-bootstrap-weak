\documentclass[11pt]{article}
\IfFileExists{VERSION.tex}{\input{VERSION}}%
{\date{Pdf compiled \today. (Version unclear; run `make VERSION.tex'.)}}
\usepackage{%
  amsfonts,
  amsmath,
  amssymb,
  amsthm,
  booktabs,
  geometry,
  graphicx,
  setspace,
  slantsc,
  tabularx,
  url,
}

\usepackage[small]{caption}
\usepackage[mmddyyyy,hhmmss]{datetime}
\usepackage[T1]{fontenc}
\usepackage[multiple]{footmisc}
\usepackage[sort,round]{natbib}

\urlstyle{same}

\date{6 October, 2014}

\bibliographystyle{abbrvnat}
\newcommand\citepos[2][]{\citeauthor{#2}'s \citeyearpar[#1]{#2}}

\newtheorem{innercustomlem}{Lemma}
\newenvironment{customlem}[1]
  {\renewcommand\theinnercustomlem{#1}\innercustomlem}
  {\endinnercustomlem}

\newtheorem{thm}{Theorem}
\newtheorem{lem}{Lemma}
\newtheorem{claim}{Claim}
\newtheorem{cor}{Corollary}
\newtheorem{res}{Result}

\theoremstyle{definition}

\newtheorem{example}{Example}
\newtheorem{defn}{Definition}
\newtheorem{rem}{Remark}

\DeclareMathOperator*{\argmin}{arg\,min}
\DeclareMathOperator*{\plim}{plim}

\DeclareMathOperator{\dist}{d}
\DeclareMathOperator{\const}{const}
\DeclareMathOperator{\cov}{cov}
\DeclareMathOperator{\E}{E}
\DeclareMathOperator{\eqd}{\overset{d}{=}}
\DeclareMathOperator{\ind}{1}
\DeclareMathOperator{\pr}{Pr}
\DeclareMathOperator{\sgn}{sgn}
\DeclareMathOperator{\var}{var}
\DeclareMathOperator{\vech}{vech}

\renewcommand{\mod}{\operatorname{mod}}

\newcommand{\clt}{\textsc{clt}}
\newcommand{\fclt}{\textsc{fclt}}
\newcommand{\hac}{\textsc{hac}}
\newcommand{\lln}{\textsc{lln}}
\newcommand{\mds}{\textsc{mds}}
\newcommand{\ned}{\textsc{ned}}

\newcommand{\textif}{\text{if}}
\newcommand{\textand}{\text{and}}

\newcommand{\sigmafield}{$\sigma$-field}

\newcommand{\Fs}{\mathcal{F}}
\newcommand{\Gs}{\mathcal{G}}
\newcommand{\Ms}{\mathcal{M}_n}
\newcommand{\Pm}{\pr_{\mathcal{M}}}
\newcommand{\Em}{\E_{\mathcal{M}}}
\newcommand{\BigC}{~\Big|~}
\newcommand{\Zs}{\mathcal{Z}}
\DeclareMathOperator{\discreteuniform}{discrete\ uniform}

\frenchspacing

\begin{document}

\author{Gray Calhoun\thanks{Economics Department, Iowa State
    University, Ames, IA 50011. Telephone: (515) 294-6271.  Email:
    \protect\url{gcalhoun@iastate.edu}. Web:
    \protect\url{http://gray.clhn.org}.}}

\title{Supplemental Appendix for ``Block bootstrap consistency under
  weak assumptions'' (Not for publication)}

\maketitle

\begin{abstract}
  \noindent This appendix contains detailed proofs of several of the
  supporting results in the paper, ``Block bootstrap consistency under
  weak assumptions.'' As we have stated elsewhere these proofs are
  fundamentally identical to existing proofs in this literature, but
  our indexing changes the presentation slightly. We are not in any
  way trying to take credit for the insight that led to these results;
  they are presented only for readers' reference and convenience.
\end{abstract}

{%
\newcommand{\isum}{\sum_{i=0}^{\lfloor n/m \rfloor - 1}}
\begin{customlem}{6}
  Suppose the conditions of Theorem~1 hold.
  For any positive $\delta$, there exist positive and finite constants
  $C$, $n_0$, and $\epsilon$ such that,
  for all $n > n_0$, $m = 1,\dots,n$, $\tau = 0,\dots,m$, and $\ell =
  1,\dots,m-1$,
  \begin{multline}\label{eq:1}\tag{37}
    \E \Big\lvert \isum \big[Z_n'(\tau + i m, m - \ell)^2
    - \E \big(Z_n'(\tau + i m, m - \ell)^2\big) \big]\Big\rvert \\
    \leq 2 \delta + C \cdot \big(\tfrac{m}{n}\big)^{1/2}
    \big(\tfrac{m}{\ell^{1+\epsilon}}\big)^{1/2}.
  \end{multline}
  Also, there exists a constant $C$ and bounded function $D(x)$ with
  $D(x) \to 0$ as $x \to \infty$ such that, for large enough $n$,
  \begin{equation}
    \label{eq:2}\tag{38}
    \E \Big\lvert \isum \E(Z_n'(\tau + i m, m - \ell)^2
    - \sigma^2 \Big\rvert \leq C\, D(\ell).
  \end{equation}
\end{customlem}
\setcounter{equation}{39}
Results~\eqref{eq:1} and~\eqref{eq:2} are direct extensions of
\citepos{Jon:97} Lemmas 5 and 4, respectively, replacing de Jong's
implicit use of inequalities with explicit inequalities. We can assume
that $\mu_{nt} = 0$ without loss of generality in these proofs.

\begin{proof}[Proof of~\eqref{eq:1}]
  Define $h(x, B) = x \ind\{|x| \leq B\} + \sgn(x) B \ind\{|x| > B\}$.
  Since Lemma~7 ensures uniform integrability, we can choose $B$ large
  enough that
  \begin{equation*}
    \E \Big| \sum_{i=0}^{\lfloor n/m \rfloor - 1} \big(
    Z'_n(\tau + im, m-\ell)^2 n / m -
    h(Z'_n(\tau + im, m-\ell), B \sqrt{m/n})^2\big)\Big| < \delta,
  \end{equation*}
  so it suffices to bound
  \begin{equation*}
    \E \Big| \sum_{i=0}^{\lfloor n/m \rfloor - 1}
    \big(h(Z'_n(\tau + im, m-\ell), B \sqrt{m/n})^2 -
    \E h(Z'_n(\tau + im, m-\ell), B \sqrt{m/n})^2\big)\Big|.
  \end{equation*}

  As in de Jong's proof,
  $h(Z'_n(\tau + im, m-\ell), B \sqrt{m / n})^2$
  is $L_2$ \ned\ with respect to
  \begin{equation*}
    \mathcal{V}_{n,i-j}^{i+j} =
    \sigma(V_{n,\tau + (i-j-1)m + \ell + 1},\dots,
             V_{n,\tau + (i+j)m}).
  \end{equation*}
  This can be seen through the sequence of inequalities (for $j > 0$):
  \begin{align*}
    \lVert h(Z'_n(\tau + im, m-\ell)&, B \sqrt{m / n})^2
    - \E(h(Z'_n(\tau + im, m-\ell), B \sqrt{m / n})^2 \mid
         \mathcal{V}_{n,i-j}^{i+j}) \rVert_2 \\
    & \leq 2 B \sqrt{\tfrac{m}{n}}
    \lVert h(Z'_n(\tau + im, m-\ell), B \sqrt{m / n}) \\
    &\quad- \E(h(Z'_n(\tau + im, m-\ell), B \sqrt{m / n}) \mid
         \mathcal{V}_{n,i-j}^{i+j}) \rVert_2 \\
    &\leq 2 B \tfrac{\sqrt{m}}{n}
    \sum_{t \in I_n(\tau + im, m - \ell)}
    \lVert X_{nt} - \E(X_{nt} \mid \mathcal{V}_{n,i-j}^{i+j}) \rVert_2 \\
    & \leq 2 B \tfrac{\sqrt{m}}{n}
    \sum_{t \in I_n(\tau + im, m - \ell)} d_{nt} v_{j\ell} \\
    &= 2 B D \tfrac{m^{3/2}}{n} \, (\ell j)^{-1/2 - \epsilon}
  \end{align*}
  for some $\epsilon > 0$ and $D \geq \max_t d_{nt}$. For $j = 0$,
  we have
  \begin{align*}
    \lVert h(Z'_n(\tau + im, m-\ell), B \sqrt{m / n})^2
    - \E(h(Z'_n(\tau + im&, m-\ell), B \sqrt{m / n})^2 \mid
         \mathcal{V}_{n,i}^{i}) \rVert_2 \\
    & \leq 2 B \sqrt{\tfrac{m}{n}} \lVert Z'_n(\tau + im, m - \ell) \rVert_2 \\
    & \leq 2 B D \tfrac{m}{n}
  \end{align*}
  with the last inequality holding by Lemma~7.

  These \ned\ inequalities further imply that $h(Z'_n(\tau + im, m -
  \ell))^2$ is an $L_2$-mixingale of size $-1/2$. Define
  \begin{equation*}
     \mathcal{H}_{nk} = \sigma(V_{n,\tau + km}, V_{n,\tau + (k-1)m},\dots)
  \end{equation*}
  For $j > 0$,
  \begin{align*}
    \lVert h(Z'_n(\tau &+ im, m - \ell, B \sqrt{m/n}))^2 -
    \E(h(Z'_n(\tau + im, m - \ell), B \sqrt{m/n})^2 \mid
    \mathcal{H}_{n,i - 2j}) \rVert_2 \\
    &\leq 
    \lVert h(Z'_n(\tau + im, m - \ell), B \sqrt{m/n})^2 \\
    &\quad -
    \E(h(Z'_n(\tau + im, m - \ell), B \sqrt{m/n})^2 \mid V_n(i + j, i-j)) \rVert_2 \\
    & \quad
    + \lVert \E(h(Z'_n(\tau + im, m - \ell), B \sqrt{m/n})^2 \mid V_n(i+ j, i-j))\\
    & \quad-
    \E(h(Z'_n(\tau + im, m - \ell), B \sqrt{m/n})^2 \mid \mathcal{H}_{n,i - 2j}) \lVert_2 \\
    &\leq 2 B D \tfrac{m^{3/2}}{n} \, (\ell j)^{-1/2 - \epsilon}
    + 2 B D \tfrac{m}{n} \psi(j \ell)
  \end{align*}
  where $\psi(x) = x^{1/2 - 1/r}$ if $V_{nt}$ is strong mixing and
  $\psi(x) = x^{1 - 1/r}$ if $V_{nt}$ is uniform mixing. By
  assumption, $\psi(x) = O(x^{-1/2 - \epsilon'})$ for some $\epsilon'
  > 0$. Assume without loss of generality that the
  $\epsilon$ we defined earlier satisfies this requirement as well,
  so $\psi(j \ell) \leq D' (j \ell)^{-1/2 - \epsilon}$ for some constant
  $D' > 1$.

  For the rest of the mixingale inequalities, let $j \geq 0$ and then
  \begin{align*}
    \lVert h(Z'_n(\tau &+ im, m - \ell, B \sqrt{m/n}))^2 -
    \E(h(Z'_n(\tau + im, m - \ell), B \sqrt{m/n})^2 \mid
    \mathcal{H}_{n,i + 2j}) \rVert_2 \\
    & \leq \lVert h(Z'_n(\tau + im, m - \ell), B \sqrt{m/n})^2 \\
    & \quad - \E(h(Z'_n(\tau + im, m - \ell), B \sqrt{m/n})^2 \mid
    V_n(i + j, i-j)) \rVert_2 \\
    & \leq 2 B D \tfrac{m^{3/2}}{n} (\ell j)^{-1/2 - \epsilon},
  \end{align*}
  completing the argument that $h(Z'_n(\tau + im, m - \ell, B
  \sqrt{m/n}))^2$ is an $L_2$-mixingale of size $-1/2$.

  Now let $C = 4 B D D'$.
  We can now apply de Jong's Lemma 2 (originally presented in
  \citealp{Mcl:75}) to this mixingale, giving
  \begin{align*}
    \Big \lVert \sum_{i=0}^{\lfloor n/m \rfloor - 1}
    \big(h(Z'_n(\tau + im, m - \ell, B \sqrt{m/n}))^2 -
    \E h(Z'_n&(\tau + im, m - \ell), B \sqrt{m/n})^2\big) \Big\rVert_2 \\
    &\leq \Bigg(\sum_{i=0}^{\lfloor n/m \rfloor - 1}
    \big(C \tfrac{m}{n} \tfrac{m^{1/2}}{\ell^{1/2 + \epsilon}}\big)^2 \Bigg)^{1/2} \\
    &\leq C \big(\tfrac{m}{n}\big)^{1/2} \tfrac{m^{1/2}}{\ell^{1/2 + \epsilon}}
  \end{align*}
  which gives the final result.
\end{proof}

\begin{proof}[Proof of~\eqref{eq:2}]
  Observe that
  \begin{align}
    \Big\lvert \notag
    \sum_{j=0}^{\lfloor n/m \rfloor - 1} & \sum_{i=j+1}^{\lfloor n/m \rfloor - 1}
    \E Z_{n}'(\tau + im, m - \ell) Z_{n}'(\tau + jm, m - \ell) \Big\rvert \\
    &\leq \tfrac{1}{n} \notag
    \sum_{j=0}^{\lfloor n/m \rfloor - 1} \sum_{i=j+1}^{\lfloor n/m \rfloor - 1}
    \sum_{t = j m + \ell + 1}^{(j+1) m} \sum_{k= i m + \ell + 1 - t}^{(i+1)m - t}
    1\{k \geq \ell\} \E\lvert X_{nt} X_{n,t+k}  \rvert\\ \label{eq:3}
    &\leq \tfrac{1}{n} \sum_{t = 1}^n \sum_{k = 0}^{n-t} 1\{k \geq \ell\}
    \sum_{v=0}^\infty  \big( \E \Delta_{1n}(t,v) \E \Delta_{2n}(t,k,v) \big)^{1/2}\\
    &\quad + \tfrac{1}{n} \sum_{t = 1}^n \sum_{k = 0}^{n-t} 1\{k \geq \ell\} \label{eq:4}
    \sum_{v=1}^\infty \big( \E \Delta_{3n}(t,v) \E \Delta_{4n}(t,k,v) \big)^{1/2}
  \end{align}
  by de Jong's Lemma 3, with
  \begin{align*}
    \Delta_{1n}(t,v) &= (\E_{t-v} X_{nt})^2 - (\E_{t-v-1} X_{nt})^2 \\
    \Delta_{2n}(t,k,v)&= (\E_{t-v}X_{n,t+k})^2 - (\E_{t-v-1} X_{n,t+k})^2 \\
    \Delta_{3n}(t,v) &=  (X_{nt} - \E_{t+v-1} X_{nt})^2 -  (X_{nt} - \E_{t+v} X_{nt})^2 \\
    \Delta_{4n}(t,k,v) &= (X_{n,t+k} - \E_{t+v-1} X_{n,t+k})^2 - (X_{n,t+k} - \E_{t+v} X_{n,t+k})^2
  \end{align*}

  Both~\eqref{eq:3} and~\eqref{eq:4} can be bounded by very similar
  arguments, so we just present the first. Now
  \begin{multline*}
    \tfrac{1}{n} \sum_{t=1}^n \sum_{k=0}^{n-t}  1\{k \geq \ell\}
    \sum_{v=0}^\infty \big(\E \Delta_{1n}(t,v) \E \Delta_{2n}(t,k,v)\big)^{1/2} \leq \\
    \sum_{v=0}^\infty \Big\{
    \Big(\tfrac{1}{n} \sum_{t=1}^n \E \Delta_{1n}(t, v) \Big)^{1/2}
    \Big(\tfrac{1}{n} \sum_{t=1}^n \Big(\sum_{k=0}^{n-t}
    1\{k \geq \ell\} (\E \Delta_{2n}(t,k,v))^{1/2} \Big)^2 \Big)^{1/2}
    \Big\}
  \end{multline*}
  by the Cauchy-Schwarz inequality.  By the same inequality, we have
  \begin{align*}
    \tfrac{1}{n} \sum_{t=1}^n \Big(\sum_{k=0}^{n-t}
    1\{k \geq \ell\} (\E & \Delta_{2n}(t,k,v))^{1/2} \Big)^2 \\
     & \leq (C/n) \sum_{t=1}^n
     \sum_{k=\ell}^{n-t} k \log(k)^2 \E \Delta_{2n}(t,k,v) \\
     & \leq (C/n) \sum_{t=1}^n c_{nt}^2 \Big(\ell \log(\ell)^2 \psi(\ell)^2
     + \sum_{k=\ell}^{\infty} \log(k)^2 \psi(k)^2 \Big)
  \end{align*}
  where $C$ is any constant that bounds $\sum_{k=1}^{\infty} k^{-1}
  \log(k)^{-2}$ and $\psi$ is defined as $\psi(k) \equiv v_k + 6 \alpha_{k/2}^{1/p -1/r}$ (if
  the underlying mixing process is strong mixing with coefficients
  $\alpha_k$) or $\psi(k) \equiv v_k + 6 \phi_{k/2}^{1 - 1/r}$ (if the
  underlying process is uniform mixing with coefficients $\phi_k$).%
\footnote{Note that these values of $\psi(k)$ are the mixingale
  indices corresponding to our \ned\ array. See \citet[Theorem 17.15]{Dav:94}
  for details.} %
  Notice that this last expression does not depend on $v$.

  Similarly,
  \begin{align*}
    \sum_{v=0}^{\infty}\Big(\tfrac{1}{n} \sum_{t=1}^n \E \Delta_{1n}(t,v) \Big)^{1/2}
    & \leq
     \Big((C/n) \sum_{t=1}^{n}\Big( \E \Delta_{1n}(t,0) + \sum_{v=1}^{\infty} \log(v)^2
     \E \Delta_{1n}(t,v)\Big) \Big)^{1/2} \\
     & \leq \Big((C/n) \sum_{t=1}^n c_{nt}^2
     \Big( \psi(0)^2 + \sum_{v=1}^{\infty} \log(v)^2 \psi(v)^2 \Big)\Big)^{1/2}.
  \end{align*}
  So we can bound~\eqref{eq:3} with
  \begin{multline*}
    \text{const} \times D(\ell) = \Big((C/n) \sum_{t=1}^{n} c_{nt}^2\Big) \\
    \times \Big(\ell \log(\ell)^2 \psi(\ell)^2
     + \sum_{k=\ell}^{\infty} \log(k)^2 \psi(k)^2 \Big)^{1/2}
     \Big( \psi(0)^2 + \sum_{v=1}^{\infty} \log(v)^2 \psi(v)^2 \Big)^{1/2}
  \end{multline*}
  and $D(\ell) \to 0$ as $\ell \to \infty$.
  A similar proof holds for~\eqref{eq:4}, so~\eqref{eq:2} holds with
  \begin{equation*}
    D(\ell) =
    \Big(\ell \log(\ell)^2 \psi(\ell)^2
    + \sum_{k=\ell}^{\infty} \log(k)^2 \psi(k)^2 \Big)^{1/2}. \qedhere
  \end{equation*}
\end{proof}
}

{%
\newcommand{\uiterm}{\max_{m \in 1,\dots,m'} \Big(
  \sum_{t \in I_n(\tau, m)} (X_{nt} - \mu_{nt})\Big)^2 \Big/
  \sum_{t\in I_n(\tau, m')} c_{nt}^2}
\newcommand{\uitermb}{\sum_{\tau=0}^{n-1} \tfrac{1}{n m'} \max_{m \in 1,\dots,m'} \Big(
  \sum_{t \in I_n(\tau, m)} (X_{nt} - \bar X_n)\Big)^2}
\begin{customlem}{7}
  Under the conditions of Theorem~1,
  \begin{multline}\label{eq:5}\tag{39}
    \lim_{C \to 0} \limsup_{n \to \infty} \sup_{\substack{\tau = 0,\dots,n-1 \\  m' = 1,\dots,n}}
    \E\big(\big(\max_{m=1,\dots,m'} Z'_n(\tau, m)^2 n / m'\big) \\
    \times\ind\{\max_{m=1,\dots,m'} Z'_n(\tau, m)^2 n / m' > C\}\big) = 0.
  \end{multline}
\end{customlem}

\begin{proof}[Proof of Lemma 7]
  \newcommand{\w}{w_{n}(\tau, m')}
  \newcommand{\sums}[1]{\Big(\sum_{t \in I_{n}(\tau, m)} #1_{nt} \Big)^2}
  \renewcommand{\uiterm}{\max_{m \in 1,\dots,m'} \Big(
    \sum_{t \in I_n(\tau, m)} X_{nt}\Big)^2 \Big/
    \Big(\sum_{t\in I_n(\tau, m')} c_{nt} \Big)^2}

  The argument follows \citet[Lemma 6.5]{Mcl:75b} and
  \citet[Lemma 3.5]{Mcl:77} almost exactly and is also presented as
  Theorem~16.13 in \citet{Dav:94}. We present the proof here to show
  that it continues to hold under our indexing strategy.

  Let $\epsilon$ be an arbitrary positive number.
  Without loss of generality, assume that $X_{nt}$ has mean zero for
  all $n$ and $t$. Define $\w^2 = \sum_{t\in I_n(\tau, m')} c_{nt}^2$
  and separate $X_{nt}$ into three components,
  \begin{align*}
    X_{nt}&= U_{nt} + T_{nt} + Y_{nt}
    \intertext{with}
    U_{nt} &= X_{nt} - \E_{t+k} X_{nt} + \E_{t-k} X_{nt} \\
    T_{nt} &= \E_{t+k} X_{nt} \ind\{|X_{nt}| > C' c_{nt}\}
             - \E_{t-k} X_{nt} \ind\{|X_{nt}| > C' c_{nt}\} \\
    Y_{nt} &= \E_{t+k} X_{nt} \ind\{|X_{nt}| \leq C' c_{nt}\}
             - \E_{t-k} X_{nt} \ind\{|X_{nt}| \leq C' c_{nt}\}
  \end{align*}
  where $k$ and $C'$ are arbitrary constants that will be constrained
  later in the proof --- this representation holds for any value of
  these constants. For convenience, define
  \begin{align*}
    x_{n}(\tau,m') &= \max_{m \in 1,\dots,m'}
                     \Big(\sum_{t \in I_n(\tau, m)} X_{nt}\Big)^2 \big/ \w^2 \\
    u_{n}(\tau,m') &= \max_{m \in 1,\dots,m'}
                     \Big(\sum_{t \in I_n(\tau, m)} U_{nt}\Big)^2 \big/ \w^2 \\
    y_{n}(\tau,m') &= \max_{m \in 1,\dots,m'}
                     \Big(\sum_{t \in I_n(\tau, m)} Y_{nt}\Big)^2 \big/ \w^2 \\
    \intertext{and}
    z_{n}(\tau,m') &= \max_{m \in 1,\dots,m'}
                     \Big(\sum_{t \in I_n(\tau, m)} T_{nt}\Big)^2 \big/ \w^2.
  \end{align*}
  Using the Cauchy-Schwarz inequality and basic algebra, we have for
  any $\tau$ and $m'$ the inequality
  \begin{equation*}
    x_n(\tau, m') \leq 3\big(u_n(\tau, m')
                       + y_n(\tau, m') + z_n(\tau, m') \big)
  \end{equation*}
  which, along with a well known inequality
  (Theorem 9.29 in \citealp{Dav:94}) gives the bounds {%
    \newcommand{\tails}[2]{\E( #1 \ind\{#1 > #2\})}
    \begin{align*}
      \E(x_n(\tau, m')&\ind\{x_n(\tau, m')> C\}) \\
      &\leq 6\big(\tails{u_n(\tau, m')}{C/6}
            + \tails{y_n(\tau, m')}{C/6}\big)\\
      &\quad + \tails{z_n(\tau, m')}{C/6} \\
      &\leq 6\big(\E u_n(\tau, m') + \tails{y_n(\tau, m')}{C/6}
      + \E z_n(\tau, m')\big)
    \end{align*}
  }
  for any positive $C$.

  Now observe that $\| \E_{t-l} U_{nt} \|_2 \leq d_{nt} v_{\max(k,l)}$
  and $\| U_{nt} - \E_{t+l} U_{nt} \|_2 \leq d_{nt} v_{\max(k,l)+1}$
  for positive $l$, making $U_{nt}$ an $L_2$-mixingale of size $-1/2$.
  Consequently, for any fixed $\tau$ and $m'$ satisfying $\tau + m'
  \leq n$, we can apply Theorem 1.6 of \cite{Mcl:75} to get the bound
  \begin{equation}\label{eq:7}
    \E u_n(\tau, m') \leq
    8 \Big((k + 1) k^{-1-\delta} + \sum_{s=k+1}^\infty s^{-1-\delta}\Big)
    \Big(v_k^2 \, k^{1+\delta} + 2 \sum_{s=k+1}^\infty v_s^2 \, s^\delta\Big),
  \end{equation}
  where $\delta > 0$ satisfies $v_k = O(k^{-1/2 - \delta})$. (This
  $\delta$ must exist because of our assumptions on the size of the
  \ned\ array.) This bound is $O(k^{-\delta})$ as $k \to \infty$
  uniformly in $n$, $m'$, and $\tau$.
  When $\tau + m' > n$, we have (from the Cauchy-Schwarz inequality)
  \begin{equation*}
    \E u_n(\tau, m') \leq
    4 \max\big( \E u_n(\tau, n - \tau),
                \E u_n(0, m' + \tau - n) \big),
  \end{equation*}
  and both terms individually satisfy~\eqref{eq:7}. As a result, we
  can choose $k$ large enough that
  \begin{equation*}
    \E u_n(\tau, m') \leq \epsilon/2
  \end{equation*}
  for all $n$, $m'$, and $\tau$.

  We can apply essentially the same argument to $z_n(\tau, m')$. For
  positive $l$, we have
  \begin{align*}
    \| \E_{t-l} T_{nt} \|_2 &=
    \big(\E[(\E_{t-\min(l, k)} T_{nt})^2 - (\E_{t-k} T_{nt})^2]\big)^{1/2} \\
    &\leq \big(\E( X_{nt}^2/c_{nt}^2
    \ind\{|X_{nt}/c_{nt}| > C'\}) \big)^{1/2}\ind\{l < k\}
  \end{align*}
  and
  \begin{align*}
    \| T_{nt} - \E_{t+l} T_{nt} \|_2
    &= \big(\E[ (\E_{t+k} T_{nt} )^2 - (\E_{t+\min(l,k)} T_{nt})^2] \big)^{1/2}\\
    &\leq \big(\E( X_{nt}^2/c_{nt}^2
    \ind\{|X_{nt}/c_{nt}| > C'\})\big)^{1/2} \ind\{l < k\}
  \end{align*}
  so $T_{nt}$ is an $L_2$-mixingale as well. Applying the same steps
  for $z_{n}(\tau, m')$ as for $u_n(\tau, m')$ gives the upper bound
  \begin{align*}
    \E z_n(\tau, m')
    &\leq 16 (k+1) \max_{t \in I_n(\tau, m')} \E\big( X_{nt}^2/c_{nt}^2
    \ind\{|X_{nt}/c_{nt}| > C'\}\big) \\
    &\leq 16 (k+1) \max_{t = 1,\dots,n} \E\big( X_{nt}^2/c_{nt}^2
    \ind\{|X_{nt}/c_{nt}| > C'\}\big)
  \end{align*}
  for any $\tau$, $m'$, and $n$ satisfying $\tau + m' \leq n$, and
  \begin{equation*}
    \E z_n(\tau, m')
    \leq 4 \max\big(\E z_n(\tau, n - \tau),
                    \E z_n(0, m' + \tau - n) \big)
  \end{equation*}
  when $\tau + m' > n$.
  Since $X_{nt}^2/c_{nt}^2$ is uniformly integrable, set $C'$ large
  enough that this upper bound is less than $\epsilon/2$ for all $n$,
  $m'$, and $\tau$ as well.

  Finally, to bound $\E(y_n(\tau,m') \ind\{y_n(\tau, m') > C/6\})$,
  use the inequality
  \begin{equation*}
    \E(y_n(\tau,m') \ind\{y_n(\tau, m') > C/6\})
    \leq 6 \E y_n(\tau, m')^2 / C.
  \end{equation*}
  We can write $Y_{nt} = \sum_{l=1-k}^k \xi_n(t,l)$, where
  \begin{equation*}
    \xi_n(t,l) = \E_{t+l} X_{nt} \ind\{|X_{nt}| \leq C' c_{nt}\}
                 - \E_{t+l-1} X_{nt} \ind\{|X_{nt}| \leq C' c_{nt}\},
  \end{equation*}
  and $\{\xi_n(t,l), \Fs_{n,t+l}\}$ forms a martingale difference
  array for each $l$. Then, when $\tau + m' \leq n$, we have
  \begin{align*}
    \E y_n(\tau, m')^2
    &= \E \max_{m = 1,\dots, m'}
    \Big| \sum_{t \in I_n(\tau, m)} \sum_{l=1-k}^k \xi_n(t,l) \Big|^4 \Big/ \w^4 \\
    &\leq \sum_{l=1-k}^k \E \max_{m = 1,\dots, m'}
    \Big| \sum_{t \in I_n(\tau, m)} \xi_n(t,l) \Big|^4 \Big/ \w^4 \\
    &\leq \frac{4^4 (2k + 1)^3}{3^4\w^4}
    \sum_{l=1-k}^k \E \Big| \sum_{t \in I_n(\tau, m')} \xi_n(t,l) \Big|^4
  \end{align*}
  where the second inequality follows from a maximal inequality for
  \mds es \citep[Theorem 16.8]{Dav:94}. Since the $\xi_n(\tau,l)$ are
  all bounded by $2 c_{nt} C'$, we can expand the expectation
  recursively to derive the bound
  \begin{equation*}
    \E \Big| \sum_{t \in I_n(\tau, m')} \xi_n(t,l) \Big|^4 \leq
    11 (2 C')^4 \w^4,
  \end{equation*}
  \citep[For details, see][Equations 16.69--16.72]{Dav:94}
  giving
  \begin{equation*}
    \E y_n(\tau, m')^2 \leq \frac{11 \times 4^6\, (2k + 1)^4 C^{\prime4}}{3^4}.
  \end{equation*}
  When $\tau + m' > n$, the same bound holds, but with $4^7$ replacing
  $4^6$.  Then, as $C \to \infty$, this quantity converges to zero,
  completing the proof of (\ref{eq:5}).
\end{proof}
}
\bibliography{texextra/references}
\end{document}

%%% Local Variables:
%%% mode: latex
%%% TeX-master: t
%%% End:
