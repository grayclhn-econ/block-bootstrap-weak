\documentclass[12pt]{article}
\frenchspacing
\usepackage{%
  amsfonts,
  amsmath,
  amssymb,
  amsthm,
  booktabs,
  geometry,
  graphicx,
  setspace,
  slantsc,
  tabularx,
  url,
}

\usepackage[small]{caption}
\usepackage[mmddyyyy,hhmmss]{datetime}
\usepackage[T1]{fontenc}
\usepackage[multiple]{footmisc}
\usepackage[sort,round]{natbib}

\urlstyle{same}

\date{6 October, 2014}

\bibliographystyle{abbrvnat}
\newcommand\citepos[2][]{\citeauthor{#2}'s \citeyearpar[#1]{#2}}

\newtheorem{innercustomlem}{Lemma}
\newenvironment{customlem}[1]
  {\renewcommand\theinnercustomlem{#1}\innercustomlem}
  {\endinnercustomlem}

\newtheorem{thm}{Theorem}
\newtheorem{lem}{Lemma}
\newtheorem{claim}{Claim}
\newtheorem{cor}{Corollary}
\newtheorem{res}{Result}

\theoremstyle{definition}

\newtheorem{example}{Example}
\newtheorem{defn}{Definition}
\newtheorem{rem}{Remark}

\DeclareMathOperator*{\argmin}{arg\,min}
\DeclareMathOperator*{\plim}{plim}

\DeclareMathOperator{\dist}{d}
\DeclareMathOperator{\const}{const}
\DeclareMathOperator{\cov}{cov}
\DeclareMathOperator{\E}{E}
\DeclareMathOperator{\eqd}{\overset{d}{=}}
\DeclareMathOperator{\ind}{1}
\DeclareMathOperator{\pr}{Pr}
\DeclareMathOperator{\sgn}{sgn}
\DeclareMathOperator{\var}{var}
\DeclareMathOperator{\vech}{vech}

\renewcommand{\mod}{\operatorname{mod}}

\newcommand{\clt}{\textsc{clt}}
\newcommand{\fclt}{\textsc{fclt}}
\newcommand{\hac}{\textsc{hac}}
\newcommand{\lln}{\textsc{lln}}
\newcommand{\mds}{\textsc{mds}}
\newcommand{\ned}{\textsc{ned}}

\newcommand{\textif}{\text{if}}
\newcommand{\textand}{\text{and}}

\newcommand{\sigmafield}{$\sigma$-field}

\newcommand{\Fs}{\mathcal{F}}
\newcommand{\Gs}{\mathcal{G}}
\newcommand{\Ms}{\mathcal{M}_n}
\newcommand{\Pm}{\pr_{\mathcal{M}}}
\newcommand{\Em}{\E_{\mathcal{M}}}
\newcommand{\BigC}{~\Big|~}
\newcommand{\Zs}{\mathcal{Z}}
\DeclareMathOperator{\discreteuniform}{discrete\ uniform}

\frenchspacing

\usepackage[tiny]{titlesec}

\begin{document}
\section*{\hfill Reply to referee 3\hfill}

In your report, you wrote,
\begin{quotation}
  \noindent\textit{``Yes, your array notation is of course needed to
    accommodate the assumption of mean heterogeneity. However, what I
    was really hoping for was some additional transparency in the form
    of an example of a mean array which would satisfy Assumption 2 of
    Theorem 1, and also perhaps a case which would fail to satisfy
    it.}

  \textit{``Array notation has commonly been used in asymptotic
    analysis to increase the generality of results, especially by
    allowing flexible scaling by sample size such as
    $X_{nt} = X_t/a_n$ where $a_n$ can be either $\sqrt{n}$ or
    something else. The object here is quite different, such cases
    would not satisfy Assumption 2 and are ruled out. It would
    therefore help the general reader, who is not inside your head at
    the outset, to explain the role of the arrays in this context,
    most helpfully by quoting an example or two. Footnote 6 is not
    helpful in this respect. Lots of people like to read papers in ET
    even when they are not particularly familiar with the precedent
    literature. Giving such readers the relevant background is just a
    courtesy. We are only talking 2 or 3 extra sentences at most.''}
\end{quotation}

Thank you for clarifying this. I have added new material that explains
when these assumptions are satisfied when there is a single break in
the mean of the stochastic process. Specifically, it explains how the
break size and break timing can affect whether the assumptions
hold. This section starts half-way through page 4 and is a bit more
than half a page long.

Working through this example revealed that the original assumption
on the means was too restrictive, so I have also relaxed the
assumption on the the dispersion of the means from
$\sum_t (\mu_{nt} - \bar \mu_n)^2) \to 0$ to
$\sum_t (\mu_{nt} - \bar \mu_n)^2) = o(n p_n^2)$ (using notation for
the stationary bootstrap; the assumption has been relaxed for all
three of the block bootstraps considered).  This change does not
affect the mathematical proofs very much: I have added a very simple
result (Lemma 3) that shows that the contribution of the $\mu_{nt}$'s to the
second moment of $\bar X^*$ remains asymptotically negligible under
this assumption and strengthened one of the components of Lemma 1.

The proofs of Lemmas 4 and 6 (which were labeled Lemmas 3 and 5 in the
last version of the paper) have to be changed a little as well to
reflect the new assumption on $\mu_{nt}$. The previous version of
Lemma 6 established that (suitably normalized) $Z_i^2$s are uniformly
integrable and used that as an intermediate result to show that the
$Z_i^{*2}$s are also u.i.\ under the right normalization. The new
version skips that intermediate step (the family of $Z_i^2s$ are no
longer u.i.) but uses components of the original proof to show that
the $Z_i^{*2}$s are u.i. directly. The relaxed assumption on the
dispersion of the $\mu_{nt}$s ensures that the blocks are short
enough that the stronger assumption continues to hold for the
$\mu_{nt}^*$s.
Changing the proof in this way essentially amounts to rearranging the
elements of the original proof and invoking the new Lemma 3.

The results in the supplemental appendix are unaffected because they
only address random variables whose means have been explicitly
subtracted out.

To help make the changes as clear as possible, I have uploaded a
pdf based on `latexdiff' that highlights the differences in the
two versions as a supplemental document. I have also made a few
spelling and wording corrections that will be apparent as well.

You also wrote,
\begin{quotation}
  \noindent%
  \textit{``Just one additional comment on the text. On page
    5, line --7, should $\mu_n^*$ read $\bar{\mu}_n^*$?''}
\end{quotation}
\noindent%
Yes, thank you, this was an error. I've changed $\mu_n^*$ to
$E^* \bar X_n^*$ to highlight the connection with Theorem 1's results.

\end{document}

%%% Local Variables:
%%% mode: latex
%%% TeX-master: t
%%% End:
