\documentclass[12pt]{article}
\frenchspacing
\usepackage{%
  amsfonts,
  amsmath,
  amssymb,
  amsthm,
  booktabs,
  geometry,
  graphicx,
  setspace,
  slantsc,
  tabularx,
  url,
}

\usepackage[small]{caption}
\usepackage[mmddyyyy,hhmmss]{datetime}
\usepackage[T1]{fontenc}
\usepackage[multiple]{footmisc}
\usepackage[sort,round]{natbib}

\urlstyle{same}

\date{6 October, 2014}

\bibliographystyle{abbrvnat}
\newcommand\citepos[2][]{\citeauthor{#2}'s \citeyearpar[#1]{#2}}

\newtheorem{innercustomlem}{Lemma}
\newenvironment{customlem}[1]
  {\renewcommand\theinnercustomlem{#1}\innercustomlem}
  {\endinnercustomlem}

\newtheorem{thm}{Theorem}
\newtheorem{lem}{Lemma}
\newtheorem{claim}{Claim}
\newtheorem{cor}{Corollary}
\newtheorem{res}{Result}

\theoremstyle{definition}

\newtheorem{example}{Example}
\newtheorem{defn}{Definition}
\newtheorem{rem}{Remark}

\DeclareMathOperator*{\argmin}{arg\,min}
\DeclareMathOperator*{\plim}{plim}

\DeclareMathOperator{\dist}{d}
\DeclareMathOperator{\const}{const}
\DeclareMathOperator{\cov}{cov}
\DeclareMathOperator{\E}{E}
\DeclareMathOperator{\eqd}{\overset{d}{=}}
\DeclareMathOperator{\ind}{1}
\DeclareMathOperator{\pr}{Pr}
\DeclareMathOperator{\sgn}{sgn}
\DeclareMathOperator{\var}{var}
\DeclareMathOperator{\vech}{vech}

\renewcommand{\mod}{\operatorname{mod}}

\newcommand{\clt}{\textsc{clt}}
\newcommand{\fclt}{\textsc{fclt}}
\newcommand{\hac}{\textsc{hac}}
\newcommand{\lln}{\textsc{lln}}
\newcommand{\mds}{\textsc{mds}}
\newcommand{\ned}{\textsc{ned}}

\newcommand{\textif}{\text{if}}
\newcommand{\textand}{\text{and}}

\newcommand{\sigmafield}{$\sigma$-field}

\newcommand{\Fs}{\mathcal{F}}
\newcommand{\Gs}{\mathcal{G}}
\newcommand{\Ms}{\mathcal{M}_n}
\newcommand{\Pm}{\pr_{\mathcal{M}}}
\newcommand{\Em}{\E_{\mathcal{M}}}
\newcommand{\BigC}{~\Big|~}
\newcommand{\Zs}{\mathcal{Z}}
\DeclareMathOperator{\discreteuniform}{discrete\ uniform}

\frenchspacing

\usepackage[tiny]{titlesec}

\begin{document}
\section*{\hfill Cover letter for the editor\hfill}

Hi Robert,

I have made the changes that referee 3 requested. In particular, I
have added new material that explains when the paper's assumptions are
satisfied when there is a single break in the mean of the stochastic
process. It explains how the break size and break timing can affect
whether the assumptions hold. This section starts half-way through
page 4 and is a bit more than half a page long.

Working through this example revealed that the original assumption
on the means was too restrictive, so I have also relaxed the
assumption on the the dispersion of the means from
$\sum_t (\mu_{NM} - \bar \mu_n)^2 \to 0$ to
$\sum_t (\mu_{nt} - \bar \mu_n)^2 = o(n p_n^2)$ (using notation for
the stationary bootstrap; the assumption has been relaxed for all
three of the block bootstraps considered).  This change does not
affect the mathematical proofs very much: I have added a very simple
result (Lemma 3) that shows that the contribution of the $\mu_{nt}$'s to the
second moment of $\bar X^*$ remains asymptotically negligible under
this assumption and strengthened one of the components of Lemma 1.

The proofs of Lemmas 4 and 6 (which were labeled Lemmas 3 and 5 in the
last version of the paper) have to be changed a little as well to
reflect the new assumption on $\mu_{nt}$. The previous version of
Lemma 6 established that (suitably normalized) $Z_i^2$s are uniformly
integrable and used that as an intermediate result to show that the
$Z_i^{*2}$s are also u.i.\ under the right normalization. The new
version skips that intermediate step (the family of $Z_i^2s$ are no
longer u.i.) but uses components of the original proof to show that
the $Z_i^{*2}$s are u.i. directly. The relaxed assumption on the
dispersion of the $\mu_{nt}$s ensures that the blocks are short
enough that the stronger assumption continues to hold for the
$\mu_{nt}^*$s.
Changing the proof in this way essentially amounts to rearranging the
elements of the original proof and invoking the new Lemma 3.

To help make the changes as clear as possible, I have uploaded a
pdf based on `latexdiff' that highlights the differences in the
two versions as a supplemental document. I have also made a few
spelling and wording corrections that will be apparent as well.
Thanks!

\strut

\noindent --- Gray Calhoun
\end{document}

%%% Local Variables:
%%% mode: latex
%%% TeX-master: t
%%% End:
