\documentclass[12pt]{article}
\usepackage[charter]{mathdesign}
\usepackage[margin=1.25in]{geometry}
\frenchspacing
\usepackage{amsmath}

\begin{document}
\hfill 2014-10-06

\noindent \textbf{Reply to referee 2}

\noindent \textbf{General summary of changes}

\begin{itemize}
\item Several terms were defined in the beginning of the proofs of
  Theorems~1 and~2, but were used repeatedly in the supporting
  lemmas. The paper now defines those terms in a separate subsection
  that proceeds the proofs.

\item I have made some of the arguments in the proof of Theorem~1 more
  explicit, and have tried to explain in detail how conditioning
  affects the intermediate steps of the proof.

\item As you noticed, the proof of stochastic equicontinuity
  was in error; it has been fixed. Fixing this proof revealed an error
  in our treatment of the expectation terms, $\mu_{nt}$, and we have
  strengthened our assumptions on their dispersion around the grand
  mean, $\bar \mu_n$. We now require
  $\sum_{t=1}^n (\mu_{nt} - \bar \mu_n) \to 0$
  as $n \to \infty$.

  However, our key assumptions on the size and moment conditions that
  the stochastic array $X_{nt}$ satisfies remain unchanged, and the
  paper's main contributions are unchanged.

\item Fixing the stochastic equicontinuity proof also led me to revise
  several of the supporting Lemmas.
  \begin{itemize}
  \item Lemmas 2 and 3 of the last draft have been replaced by Lemmas
    5 and 7. These results are new maximal inequalities.
  \item Lemma 5 of the last draft has been split up; parts of it are
    in Lemma 3 and the remainder is in Lemma 4. Moreover, Lemma 4 of
    the last draft has been incorporated into Lemma 3 of the latest draft.
  \item Lemma 6 of the last draft has been rewritten to use
    unconditional expectations instead of conditional expectations.
  \end{itemize}

\item There is some concern that the inequalities referenced in the
  paper may fail with as complicated of an indexing scheme as used in
  the paper. To address this, I have added a supplemental appendix
  that provides more detail and shows how the original results apply
  to our setting. These results are presented in the main paper as
  Lemmas 6 and 7 of the paper and the supplemental appendix contains
  their proofs. The proof are essentially identical to existing
  results in the literature, and I am not claiming any authorship of
  those particular results. But some readers will probably be
  interested in this explicit demonstration.

\item Other minor changes have been made to address points raised by
  the referees.
\end{itemize}

\newpage

\noindent \textbf{Reply to specific points}

\begin{enumerate}

\item \emph{Setup of Theorem 1: it is not clear to me from the text
    how exactly the last block that ranges up to $n$ is dealt with. Is
    $K_{n,J_n} = n$?}

  Yes, thank you for pointing this out. I've clarified that $K_{n,J_n}
  = n$.

\item \emph{Page 7, Equations (16), (17) and (18): I do not understand
    why (16) is implied by Hall and Heyde's CLT by (17) and (18). This
    may be due to my lack of exposure to the bootstrap literature. It
    seems that the author applies a conditional version of a CLT that
    is unconditional; however, the exact argument escapes me. I seem
    to be able to derive the author's result by working directly with
    Hall and Heyde's proof, but I failed to see how the shortcut that
    the author uses works. Some clarification or a good reference may
    be needed here.}

  (These equations are now numbered (21) -- (23))

  I have added some clarification and a reference to the paper; the
  key idea is that the only stochastic components of this conditional
  probability measure are the periods where each block starts; they
  have a discrete uniform distribution over $1,\dots,n$ and are
  independent of the other random variables. This ensures that the
  conditional probability measures are regular and arguments like Hall
  and Heyde's, which implicitly allow the probability space to change
  with $n$, hold without modification. (If it were an irregular
  conditional probability, this would be problematic.)

  Note that Hall and Heyde use the same approach themselves in Remark
  (vi) on p. 63 (for example) in the context of random norming ---
  applying the CLT while conditioning on an auxiliary r.v.

\item \emph{Equation below equation (19): I cannot quite see
    this. There is the issue of the last block, but that must be
    fixable because it is asymptotically negligible; but I also do not
    understand how this formula works overall. Obviously by (12),
    \begin{equation*}
      E^*(g(Z_{nj}^* \mid \mathcal{M}_n) =
      E^*\Big(g\Big(\tfrac{1}{\sqrt{n}} \sum_{t = K_{n,j-1}+1}^{K_{nj}} (X_{nt}^* - \bar X_n)\Big)
      \ \big|\ \mathcal{M}_n\Big)
    \end{equation*}
    and by (14) and (15),
    \begin{align*}
      \tfrac{1}{n} \sum_{\tau = 0}^{n-1} g(Z_n(\tau, M_{nj}))
      &= \tfrac1n \sum_{\tau = 0}^{n-1} g\Big(\tfrac1{\sqrt{n}} \sum_{t \in I_n(\tau, M_{nj})} (X_{nt} - \mu)\Big) \\
      &\approx \tfrac1n \sum_{\tau = 0}^{n-1} g\Big(\tfrac1{\sqrt{n}} \sum_{t = \tau+1}^{\tau + M_{nj}} (X_{nt} - \mu)\Big)
    \end{align*}
    where the $\approx$ indicates that I have ignored the last
    block. However, I could not see why both expressions are
    identical. One odd aspect of the last formula is that it will
    refer to values of $X_{nt}$ for $t > n$ for values of $j$ for
    which $M_{nj} \geq 1$ (this can be seen by choosing $\tau=n-1$).}

  (This equation is now numbered (18))

  $Z_n(\tau, M_{nj})$ should be defined as $\tfrac{1}{\sqrt{n}}
  \sum_{t\in I_n(\tau, m)} (X_{nt} - \bar X_n)$ --- I have corrected
  the definition in the paper. Other uses of $Z_n(\tau, M_{nj})$ in
  the paper had implicitly used the correct definition, fortunately.

  These expressions are now identical because
  \begin{equation*}
    (X_{n, K_{n,j-1}+1}^*,\dots, X_{n, K_{n,j}}^*) =
    \begin{cases}
      (X_{n,1},\dots,X_{n,M_{nj}}) & \text{with probability }1/n \\
      (X_{n,2},\dots,X_{n,M_{nj}+1}) & \text{with probability }1/n \\
      \vdots \\
      (X_{n,n}, X_{n,1}\dots,X_{n,M_{nj}-1}) & \text{with probability }1/n
    \end{cases}
  \end{equation*}

  The ``odd aspect of the last formula'' comes from your ``$\approx$''
  substitution; $I_n(\tau,m)$ is defined in the paper to wrap around
  after the $n$th observation. I.e., when $\tau = n-1$ the index
  defined by $I_n(\tau, m)$ is $(n, 1, 2,\dots, m - 1)$. Since this is
  not obvious, I've now emphasized this aspect of $I$: see Equation~(10).

  Since this argument is so important to my results, I've moved it out
  of the proof of Theorem 1 and discuss it before either proof. I have
  also stated it in a somewhat more general form (i.e. it allows for
  several blocks instead of just one) so that its use in Theorem 2 and
  the supporting lemmas are more clear.

\item \emph{p. 8, middle: ``by definition:'' perhaps explain why the
    expression is $O_{a.s.}(1)$ here.}

  Thank you for alerting me to this source of confusion. I have
  simplified some of the notation in this argument added intermediate
  steps. The relevant line is now between Equations (24) and (25).

\item \emph{p. 8: reference to Van der Vaart's Lemma 2.11: this lemma
    involves convergence of a distribution to a continuous
    distribution, while your setting involves convergence in
    probability. Therefore, perhaps you should indicate how exactly
    the extension can be made to your setting.}

  I have added more details to show how the argument applies in this
  setting.

\item \emph{Lemma 3: usually, uniform integrability of Y n is defined as
    \begin{equation*}
      \limsup_{K \to \infty} \limsup_{n \to \infty} E|X_n(\tau,m)| I(|Y_n| > K) = 0.
    \end{equation*}
    Are you claiming here that for the $X_n(τ, m)$ of Lemma 3, we have
    \begin{equation*}
      \limsup_{K \to \infty} \limsup_{\tau\to\infty} \limsup_{m \to \infty} \limsup_{n \to \infty}
      E|X_n(\tau, m)|I(|X_n(\tau, m)| > K) = 0?
    \end{equation*}
    If so, it is not clear to me that the single-index results from
    Davidson and de Jong have been translated into the triple index
    result needed here; that is, I have trouble with the last sentence
    of the proof of Lemma 3.}

  I have written a supplemental appendix that reproduces some
  inequalities and uniform integrability results. They are referenced
  as Lemmas 6 and 7 in the main text. As you can see, the triple index
  that my paper needs makes the notation more cumbersome, but does not
  change the proof in a meaningful way.

  Lemma 3 has been removed and has been replaced by an analogous but
  somewhat stronger result in Lemma 5.

\item \emph{Start of Theorem 2: I do not see why Lemma 5 has the
    stated implication.}

  I have rephrased this introduction.

\item \emph{p. 9, bottom: this result is an error. Davidson's theorem
    15.14 is a standard maximal inequality in the spirit of
    Kolmogorov's inequality. Therefore, it cannot deliver the stated
    result, because a maximal inequality such as 15.14 only maximizes
    over one index only.}

  Thank you for pointing out this embarrassing error; I have corrected
  the proof.

\item \emph{Lemma 6: this result is obtained from direct reference to
    existing proofs. Therefore as far as I can see, the proof as
    stated leaves open the possibility that $C$, $N$ and $\epsilon$
    depend on $\mathcal{M}_n$.}

  I have changed the statement of these results to rule out that
  possibility, thank you for calling my attention to it.

\end{enumerate}

\end{document}
