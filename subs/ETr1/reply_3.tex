\documentclass[12pt]{article}
\usepackage[charter]{mathdesign}
\usepackage[margin=1.25in]{geometry}

\begin{document}
\hfill 2014-10-06

\noindent \textbf{Reply to referee 3}

\noindent \textbf{General summary of changes}

\begin{itemize}
\item Several terms were defined in the beginning of the proofs of
  Theorems~1 and~2, but were used repeatedly in the supporting
  lemmas. The paper now defines those terms in a separate subsection
  that proceeds the proofs.

\item I have made some of the arguments in the proof of Theorem~1 more
  explicit, and have tried to explain in detail how conditioning
  affects the intermediate steps of the proof.

\item One referee noticed that the proof of stochastic equicontinuity
  was in error; it has been fixed. Fixing this proof revealed an error
  in our treatment of the expectation terms, $\mu_{nt}$, and we have
  strengthened our assumptions on their dispersion around the grand
  mean, $\bar \mu_n$. We now require
  $\sum_{t=1}^n (\mu_{nt} - \bar \mu_n) \to 0$
  as $n \to \infty$.

  However, our key assumptions on the size and moment conditions that
  the stochastic array $X_{nt}$ satisfies remain unchanged, and the
  paper's main contributions are unchanged.

\item Fixing the stochastic equicontinuity proof also led me to revise
  several of the supporting Lemmas.
  \begin{itemize}
  \item Lemmas 2 and 3 of the last draft have been replaced by Lemmas
    5 and 7. These results are new maximal inequalities.
  \item Lemma 5 of the last draft has been split up; parts of it are
    in Lemma 3 and the remainder is in Lemma 4. Moreover, Lemma 4 of
    the last draft has been incorporated into Lemma 3 of the latest draft.
  \item Lemma 6 of the last draft has been rewritten to use
    unconditional expectations instead of conditional expectations.
  \end{itemize}

\item There is some concern that the inequalities referenced in the
  paper may fail with as complicated of an indexing scheme as used in
  the paper. To address this, I have added a supplemental appendix
  that provides more detail and shows how the original results apply
  to our setting. These results are presented in the main paper as
  Lemmas 6 and 7 of the paper and the supplemental appendix contains
  their proofs. The proof are essentially identical to existing
  results in the literature, and I am not claiming any authorship of
  those particular results. But some readers will probably be
  interested in this explicit demonstration.

\item Other minor changes have been made to address points raised by
  the referees.
\end{itemize}

\newpage

\noindent \textbf{Reply to specific points}

\begin{description}
\item[Main comment] \emph{My main comment on the paper relates to the
    array notation used throughout. Specifically, what is it for?}

  I agree with several of the points raised here. The original array,
  $\{X_{nt}\}$, is not normalized and so the notation might benefit
  from using $X_t$ instead of $X_{nt}$. But the theoretical results
  deal with arrays like $\{Z_{nj}\}$. These arrays are normalized and
  also depend on the bootstrap lengths $M_{nj}$, which also depend on
  $n$. Ultimately, I feel that it would be more confusing to the
  reader if the paper were to use array notation for the various
  $Z_{nj}$ terms --- where it does seem helpful --- but avoid it for
  the $X_t$ terms.

  I also think that it is useful to make the normalization explicit in
  calculating the variance, etc., because it is natural to define the
  bootstrap in terms of the original $X_{nt}$s and not a normalized or
  demeaned array.

  I realize that both of these choices make the main text somewhat
  more cumbersome than it would be otherwise, but I hope that they
  make the mathematics less ambiguous.

\end{description}
\begin{enumerate}
\item[1.] \emph{In the statement of Lemma 2 (p. 13) $\mathcal{G}_n$ is a
    sigma-field of events, so in what sense are $c_{nt}$ (scalar
    constant?)  and $\Delta_n$ (real random variable?) elements of
    $\mathcal{G}_n$ ? I think something else is intended here?}

  I have rewritten the inequalities that depended on this lemma so
  that they do not require it, so this lemma has been removed from the
  paper. Originally, the notation was meant to convey that both
  $c_{nt}$ and $\Delta_n$ are $\mathcal{G}_n$-measurable random
  variables.

\item[2.] \emph{It's customary in the this type of work to define the
    probability space in which the action takes place, say $(\Sigma,
    \mathcal{F}, P)$, so that subsequently defined sigma fields can be
    identified as subsets of $\mathcal{F}$. I think that being
    explicit about this is a convention worth adhering to.}

  I agree that there are several areas of potential confusion in this
  paper. With respect to the sigma fields involved, many of our
  results rely crucially on adding different information sets at
  different steps of the analysis, often after invoking the L.I.E.;
  for example, conditioning on the original data set and the random
  block lengths, but not on the starting position of the blocks; or
  conditioning on the block lengths, but not on the original data set
  or the block starting positions; etc. I don't see a way to make
  those arguments more transparent by explicitly introducing the
  product of the bootstrap mechanism and the DGP that generated the
  original dataset as a probability space.

  However, I am very concerned about these issues and I have written a
  separate section that proceeds the proofs and defines many of the
  derived random variables and sub sigma-fields that are used in our
  proofs. Hopefully that helps clear up some of the readers' potential
  confusion.

\end{enumerate}

\end{document}
