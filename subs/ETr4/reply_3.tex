\documentclass[12pt]{article}
\frenchspacing
\usepackage{%
  amsfonts,
  amsmath,
  amssymb,
  amsthm,
  booktabs,
  geometry,
  graphicx,
  setspace,
  slantsc,
  tabularx,
  url,
}

\usepackage[small]{caption}
\usepackage[mmddyyyy,hhmmss]{datetime}
\usepackage[T1]{fontenc}
\usepackage[multiple]{footmisc}
\usepackage[sort,round]{natbib}

\urlstyle{same}

\date{6 October, 2014}

\bibliographystyle{abbrvnat}
\newcommand\citepos[2][]{\citeauthor{#2}'s \citeyearpar[#1]{#2}}

\newtheorem{innercustomlem}{Lemma}
\newenvironment{customlem}[1]
  {\renewcommand\theinnercustomlem{#1}\innercustomlem}
  {\endinnercustomlem}

\newtheorem{thm}{Theorem}
\newtheorem{lem}{Lemma}
\newtheorem{claim}{Claim}
\newtheorem{cor}{Corollary}
\newtheorem{res}{Result}

\theoremstyle{definition}

\newtheorem{example}{Example}
\newtheorem{defn}{Definition}
\newtheorem{rem}{Remark}

\DeclareMathOperator*{\argmin}{arg\,min}
\DeclareMathOperator*{\plim}{plim}

\DeclareMathOperator{\dist}{d}
\DeclareMathOperator{\const}{const}
\DeclareMathOperator{\cov}{cov}
\DeclareMathOperator{\E}{E}
\DeclareMathOperator{\eqd}{\overset{d}{=}}
\DeclareMathOperator{\ind}{1}
\DeclareMathOperator{\pr}{Pr}
\DeclareMathOperator{\sgn}{sgn}
\DeclareMathOperator{\var}{var}
\DeclareMathOperator{\vech}{vech}

\renewcommand{\mod}{\operatorname{mod}}

\newcommand{\clt}{\textsc{clt}}
\newcommand{\fclt}{\textsc{fclt}}
\newcommand{\hac}{\textsc{hac}}
\newcommand{\lln}{\textsc{lln}}
\newcommand{\mds}{\textsc{mds}}
\newcommand{\ned}{\textsc{ned}}

\newcommand{\textif}{\text{if}}
\newcommand{\textand}{\text{and}}

\newcommand{\sigmafield}{$\sigma$-field}

\newcommand{\Fs}{\mathcal{F}}
\newcommand{\Gs}{\mathcal{G}}
\newcommand{\Ms}{\mathcal{M}_n}
\newcommand{\Pm}{\pr_{\mathcal{M}}}
\newcommand{\Em}{\E_{\mathcal{M}}}
\newcommand{\BigC}{~\Big|~}
\newcommand{\Zs}{\mathcal{Z}}
\DeclareMathOperator{\discreteuniform}{discrete\ uniform}

\frenchspacing

\usepackage[tiny]{titlesec}

\begin{document}
\section*{\hfill Reply to referee 3\hfill}

\begin{enumerate}
\item \emph{$\pi$ or $c$? Which is it?} (page 4, lines --10 and --1)

It should be $\pi$. Thanks for catching this error.

\item \emph{``independent \underline{and} can\dots''} (Page 5 line --3)

Fixed, thanks.

\item \emph{Should $\Em(\cdot)$ read $\Em^*(\cdot)$ here? Similarly
    for $\Em(\cdot)$ and $\Pm(\cdot)$ on the next line.}

  Yes, for $\Em(\cdot)$. The other terms look correct, though.

  \emph{Just to clarify:
    \begin{itemize}
    \item $\Em^*(\cdot)$ denotes the expectation over the $u$, holding
      sample and the $M_{ni}$ fixed?
    \item $\Em(\cdot)$ denotes the expectation over sample and
      $u_{ni}$, holding only $M_{ni}$ fixed?
    \item $\E^*(\cdot)$ denotes the expectation over $u_{ni}$ and
      $M_{ni}$, holding sample fixed?
    \end{itemize} 
    (Ditto for the probabilities.) This could all be spelt out more
    clearly.}  (Page 7. Line --6)

  This is correct. I have also added your clarification to the paper
  (if you don't mind, of course) to help other readers see these
  essential differences.

\item \emph{``any function $g$\dots.'' Are there no restrictions of any
    kind on $g$? If so, this deserves comment.} (Page 9)

  I have added a disclaimer. Thanks.

\item \emph{I assume $\Em \Em^* = \Em$, law of iterated
    expectations. Yes?  Same on the following page.} (Page 10, equation (25))

  Yes; this part of the proof has been rewritten in response to
  another referee's comment, however.

\item \emph{``Attributed to Polya ...''. If this is worth saying, it
    is worth providing a citation.} (Page 11, line --5)

  Ok. My primary motivation in writing that statement was to avoid
  appearing to take credit for a standard argument in this
  literature. (I was asked to include these steps explicitly by an
  earlier referee.)  I have added a reference to van der Vaart's
  (2000) textbook instead, since I've been unable to track down a
  specific citation for Poly\'a.


\item \emph{Having $n$ on both sides of the limit is untidy, to say
    the least. If what is meant is that the two functions of n have
    the same limit, as it appears, there must be a neater way to write
    this?} (Page 14, lines --8 and --7)

  This first result has been cleaned up a little bit.

\item \emph{Since these arguments are nonnegative (they're
    probabilities), what is the purpose of enclosing them in absolute
    value bars?} (Page 15, line --5)

  They are enclosed in absolute value bars to make the connection to the
  $L_{1}$-norm explicit. I've removed the absolute value sign from the first
  probability to reduce confusion.

\end{enumerate}

\end{document}

%%% Local Variables:
%%% mode: latex
%%% TeX-master: t
%%% End:
