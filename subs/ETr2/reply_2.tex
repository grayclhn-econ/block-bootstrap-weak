\documentclass[12pt]{article}
\frenchspacing
\usepackage{%
  amsfonts,
  amsmath,
  amssymb,
  amsthm,
  booktabs,
  geometry,
  graphicx,
  setspace,
  slantsc,
  tabularx,
  url,
}

\usepackage[small]{caption}
\usepackage[mmddyyyy,hhmmss]{datetime}
\usepackage[T1]{fontenc}
\usepackage[multiple]{footmisc}
\usepackage[sort,round]{natbib}

\urlstyle{same}

\date{6 October, 2014}

\bibliographystyle{abbrvnat}
\newcommand\citepos[2][]{\citeauthor{#2}'s \citeyearpar[#1]{#2}}

\newtheorem{innercustomlem}{Lemma}
\newenvironment{customlem}[1]
  {\renewcommand\theinnercustomlem{#1}\innercustomlem}
  {\endinnercustomlem}

\newtheorem{thm}{Theorem}
\newtheorem{lem}{Lemma}
\newtheorem{claim}{Claim}
\newtheorem{cor}{Corollary}
\newtheorem{res}{Result}

\theoremstyle{definition}

\newtheorem{example}{Example}
\newtheorem{defn}{Definition}
\newtheorem{rem}{Remark}

\DeclareMathOperator*{\argmin}{arg\,min}
\DeclareMathOperator*{\plim}{plim}

\DeclareMathOperator{\dist}{d}
\DeclareMathOperator{\const}{const}
\DeclareMathOperator{\cov}{cov}
\DeclareMathOperator{\E}{E}
\DeclareMathOperator{\eqd}{\overset{d}{=}}
\DeclareMathOperator{\ind}{1}
\DeclareMathOperator{\pr}{Pr}
\DeclareMathOperator{\sgn}{sgn}
\DeclareMathOperator{\var}{var}
\DeclareMathOperator{\vech}{vech}

\renewcommand{\mod}{\operatorname{mod}}

\newcommand{\clt}{\textsc{clt}}
\newcommand{\fclt}{\textsc{fclt}}
\newcommand{\hac}{\textsc{hac}}
\newcommand{\lln}{\textsc{lln}}
\newcommand{\mds}{\textsc{mds}}
\newcommand{\ned}{\textsc{ned}}

\newcommand{\textif}{\text{if}}
\newcommand{\textand}{\text{and}}

\newcommand{\sigmafield}{$\sigma$-field}

\newcommand{\Fs}{\mathcal{F}}
\newcommand{\Gs}{\mathcal{G}}
\newcommand{\Ms}{\mathcal{M}_n}
\newcommand{\Pm}{\pr_{\mathcal{M}}}
\newcommand{\Em}{\E_{\mathcal{M}}}
\newcommand{\BigC}{~\Big|~}
\newcommand{\Zs}{\mathcal{Z}}
\DeclareMathOperator{\discreteuniform}{discrete\ uniform}

\frenchspacing

\usepackage[tiny]{titlesec}

\begin{document}
\section*{\hfill Reply to referee 2\hfill}

\section*{General summary of changes}
This version of the paper has the following changes from the
previous submission:
\input{summary}

\section*{Reply to specific points}

\begin{enumerate}
\item \textit{Corollary 1: Assuming there exists an estimator
$\hat{\sigma}_n^2(\gamma)$ such that $\sup_{\gamma}
|\sigma_n^2(\gamma)/\hat{\sigma}_n^2(\gamma) - 1| \to^p 0$ may not be
innocent. After all, for an i.i.d.\ sequence of mean zero, finite
variance $\varepsilon_i$, we could have
\[
\hat{\sigma}_{n}^2(\gamma) = \tfrac{1}{n} \sum_{i=1}^{[\gamma n]} \varepsilon_i^2
\quad\text{and}\quad
\sigma_n^2(\gamma) = [\gamma n] \sigma^2 / n.
\]
Then
\begin{align*}
\sup_{0 \leq \gamma \leq 1} | \sigma_n^2(\gamma) / \hat{\sigma}_n^2(\gamma) - 1 |
& = \sup_{0 \leq \gamma \leq 1} \Big| \sigma^2 \Big/ ([n\gamma])^{-1} \sum_{i=1}^{[\gamma n]} \varepsilon_i^2 - 1 \Big| \\
& = \sup_{k:1 \leq k \leq n} \Big| \sigma^2 \Big/ k^{-1} \sum_{i=1}^{k} \varepsilon_i - 1 \Big|
\end{align*}
and the last expression converges a.s. to a non-degenerate random
variable. It is not clear to me what exactly the author wants to
achieve by dividing by $\hat{\sigma}_n(\gamma)$. Clearly such a
division achieves that the resulting process is $N(0,1)$ and not
$N(0,\gamma)$ for each $\gamma$ which means that the limit is not
Wiener measure, and therefore I did not understand how the resulting
process can end up at $W^{*}$, which is undefined, but is suggested
through the use of the letter $W$ to be some type of Wiener measure.}

\item \textit{p. 11, first equation: there is an expectation of an
    absolute value there; should the absolute value be there?}

\item p. 12, Lemma 1: typo: ``$\sum_{i=1}^{J_n-1} < n$'' should read
  ``$\sum_{i=1}^{J_n-1} M_{ni} < n$''.

\item \textit{On page 7, I could not verify that (18) holds; that is,
    I could not see that none of the $X_{nt}^*$ is not counted twice
    or excluded in the $\sum_{j=1}^{J_n} Z_{nj}^*$.}

\item \textit{At this time, I still do not understand the reasoning of
    the author when it comes to conditioning issues. The author to his
    credit has added explanations, and very likely the problem is
    simply my lack of experience with bootstrap issues. However, on
    page 9 for example, the author says that Lemma 5 implies that
    there exists a finite, monotone function $B(\cdot)$ that has a
    limit of 0. As I read Lemma 5, I would think that this $B(\cdot)$
    depends on $\mathcal{M}$. Similarly, on page 15, the author
    applies Lemma 6, Equation 37; yet that result was derived for an
    unconditional expectation appearing on the spot where the author
    now uses $E_{\mathcal{M}}$. In short, I am still confused by some
    of the conditioning issues, but am willing to assume that the
    problem is with me and not with the author.}

\end{enumerate}
\end{document}
