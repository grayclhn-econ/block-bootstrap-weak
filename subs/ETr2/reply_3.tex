\documentclass[12pt]{article}
\frenchspacing
\usepackage{%
  amsfonts,
  amsmath,
  amssymb,
  amsthm,
  booktabs,
  geometry,
  graphicx,
  setspace,
  slantsc,
  tabularx,
  url,
}

\usepackage[letterspace=20]{microtype}
\usepackage[small]{caption}
\usepackage[mmddyyyy,hhmmss]{datetime}
\usepackage[T1]{fontenc}
\usepackage[multiple]{footmisc}
\usepackage[sort,round]{natbib}

\urlstyle{same}

\bibliographystyle{abbrvnat}
\newcommand\citepos[2][]{\citeauthor{#2}'s \citeyearpar[#1]{#2}}

\newtheorem{innercustomlem}{Lemma}
\newenvironment{customlem}[1]
  {\renewcommand\theinnercustomlem{#1}\innercustomlem}
  {\endinnercustomlem}

\newtheorem{thm}{Theorem}
\newtheorem{lem}{Lemma}
\newtheorem{claim}{Claim}
\newtheorem{cor}{Corollary}
\newtheorem{res}{Result}

\theoremstyle{definition}

\newtheorem{example}{Example}
\newtheorem{defn}{Definition}
\newtheorem{rem}{Remark}

\DeclareMathOperator*{\argmin}{arg\,min}
\DeclareMathOperator*{\plim}{plim}

\DeclareMathOperator{\dist}{d}
\DeclareMathOperator{\const}{const}
\DeclareMathOperator{\cov}{cov}
\DeclareMathOperator{\E}{E}
\DeclareMathOperator{\eqd}{\overset{d}{=}}
\DeclareMathOperator{\ind}{1}
\DeclareMathOperator{\pr}{Pr}
\DeclareMathOperator{\sgn}{sgn}
\DeclareMathOperator{\var}{var}
\DeclareMathOperator{\vech}{vech}

\renewcommand{\mod}{\operatorname{mod}}

\newcommand{\allcaps}[1]{\textls{\MakeUppercase{#1}}}
\newcommand{\clt}{\allcaps{clt}}
\newcommand{\fclt}{\allcaps{fclt}}
\newcommand{\hac}{\allcaps{hac}}
\newcommand{\lln}{\allcaps{lln}}
\newcommand{\mds}{\allcaps{mds}}
\newcommand{\ned}{\allcaps{ned}}

\newcommand{\textif}{\text{if}}
\newcommand{\textand}{\text{and}}

\newcommand{\sigmafield}{$\sigma$-field}

\newcommand{\Fs}{\mathcal{F}}
\newcommand{\Gs}{\mathcal{G}}
\newcommand{\Ms}{\mathcal{M}_n}
\newcommand{\Pm}{\pr_{\mathcal{M}}}
\newcommand{\Em}{\E_{\mathcal{M}}}
\newcommand{\BigC}{~\Big|~}
\newcommand{\Zs}{\mathcal{Z}}
\DeclareMathOperator{\discreteuniform}{discrete\ uniform}

\frenchspacing

\usepackage[tiny]{titlesec}

\begin{document}
\section*{\hfill Reply to referee 3\hfill}
\section*{General summary of changes}
This version of the paper has the following changes from the
previous submission:
\input{summary}

\section*{Reply to specific points}

\begin{enumerate}
\item \textit{The first [issue] concerned the use of array notation,
    in contexts where it appeared to me unnecessary. I asked
    specifically, either for this to be dropped, or for examples to be
    given showing its role in applications.}

  The aim of the paper is to relax the conditions required by
  \cite{GoW:02} and \cite{GoJ:03}. A major focus of \cite{GoW:02} is
  proving consistency of these block bootstraps under some degree of
  heterogeneity in the means\,---\,one example is, as they put it, ``a
  finite number of properly behaved breaks in means.'' For our results
  to be comparable to theirs in this way, even to show that we do not
  allow as much heterogeneity in the means as those papers do, it's
  important that we use array notation. So, an example where array
  notation is important in applciations is when there is a break in
  the mean at observation $n/2$.

  While I agree with what you wrote in the previous letter, ``it is
  never a good practice to copy what other authors do, as an
  alternative to justifying it,'' I also think that readers generally
  benefit when authors reuse decent notation from other closely
  related papers, even if a different notation is incrementally
  better.

  Another question raised in your previous letter was: why explicitly
  normalize by $1/\sqrt{n}$ and why take means? As mentioned in the
  previous paragraph, each $X_{nt}$ is allowed to have nonzero mean,
  so estimates of the variance need to be explicitly demeaned. Making
  the normalization explicit makes these operations transparent. Once
  the series is demeaned (either using the sample mean or the
  population mean), all of the mathematics uses the various $Z_{nj}$
  terms which have an implicit normalization.

  In any case, I have added an explanation for these choices as
  Footnote 6 in the paper.

\item \textit{The second [issue] concerned the specification of the
    statistical model, in particular, to define the probability space
    appropriately, so as to develop the formal relationships between
    the various sub-sigma fields, and sequences and filtrations of
    same. For example, we need a formal context for objects like the
    sequence defined in equation (9). Clarity and formalism in these
    matters are especially important where random events generated by
    the bootstrap and by nature interact.}

  I have added a formal specification of the top-level probability
  space at the beginning of the mathematical appendix. You are
  right, this has made it easier to make precise statements in some
  of the results\,---\,e.g. the statement of Lemma~2 and the proof
  of Lemma~5. Thank you for this suggestion.  I will also point out
  that the paper defines several important sub-sigma-fields as they
  are necessary for the proofs, which gives additional clarity and
  formalism, and I have tried to improve in this respect in this
  version of the paper.

\end{enumerate}

\bibliography{../../texextra/references}
\end{document}
