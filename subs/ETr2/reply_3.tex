\documentclass[12pt]{article}
\frenchspacing
\usepackage{%
  amsfonts,
  amsmath,
  amssymb,
  amsthm,
  booktabs,
  geometry,
  graphicx,
  setspace,
  slantsc,
  tabularx,
  url,
}

\usepackage[small]{caption}
\usepackage[mmddyyyy,hhmmss]{datetime}
\usepackage[T1]{fontenc}
\usepackage[multiple]{footmisc}
\usepackage[sort,round]{natbib}

\urlstyle{same}

\date{6 October, 2014}

\bibliographystyle{abbrvnat}
\newcommand\citepos[2][]{\citeauthor{#2}'s \citeyearpar[#1]{#2}}

\newtheorem{innercustomlem}{Lemma}
\newenvironment{customlem}[1]
  {\renewcommand\theinnercustomlem{#1}\innercustomlem}
  {\endinnercustomlem}

\newtheorem{thm}{Theorem}
\newtheorem{lem}{Lemma}
\newtheorem{claim}{Claim}
\newtheorem{cor}{Corollary}
\newtheorem{res}{Result}

\theoremstyle{definition}

\newtheorem{example}{Example}
\newtheorem{defn}{Definition}
\newtheorem{rem}{Remark}

\DeclareMathOperator*{\argmin}{arg\,min}
\DeclareMathOperator*{\plim}{plim}

\DeclareMathOperator{\dist}{d}
\DeclareMathOperator{\const}{const}
\DeclareMathOperator{\cov}{cov}
\DeclareMathOperator{\E}{E}
\DeclareMathOperator{\eqd}{\overset{d}{=}}
\DeclareMathOperator{\ind}{1}
\DeclareMathOperator{\pr}{Pr}
\DeclareMathOperator{\sgn}{sgn}
\DeclareMathOperator{\var}{var}
\DeclareMathOperator{\vech}{vech}

\renewcommand{\mod}{\operatorname{mod}}

\newcommand{\clt}{\textsc{clt}}
\newcommand{\fclt}{\textsc{fclt}}
\newcommand{\hac}{\textsc{hac}}
\newcommand{\lln}{\textsc{lln}}
\newcommand{\mds}{\textsc{mds}}
\newcommand{\ned}{\textsc{ned}}

\newcommand{\textif}{\text{if}}
\newcommand{\textand}{\text{and}}

\newcommand{\sigmafield}{$\sigma$-field}

\newcommand{\Fs}{\mathcal{F}}
\newcommand{\Gs}{\mathcal{G}}
\newcommand{\Ms}{\mathcal{M}_n}
\newcommand{\Pm}{\pr_{\mathcal{M}}}
\newcommand{\Em}{\E_{\mathcal{M}}}
\newcommand{\BigC}{~\Big|~}
\newcommand{\Zs}{\mathcal{Z}}
\DeclareMathOperator{\discreteuniform}{discrete\ uniform}

\frenchspacing

\usepackage[tiny]{titlesec}

\begin{document}
\section*{\hfill Reply to referee 3\hfill}
\section*{General summary of changes}
This version of the paper has the following changes from the
previous submission:
\input{summary}

\section*{Reply to specific points}

\begin{enumerate}
\item \textit{The first [issue] concerned the use of array notation,
    in contexts where it appeared to me unnecessary. I asked
    specifically, either for this to be dropped, or for examples to be
    given showing its role in applications.}

\item \textit{The second [issue] concerned the specification of the
    statistical model, in particular, to define the probability space
    appropriately, so as to develop the formal relationships between
    the various sub-sigma fields, and sequences and filtrations of
    same. For example, we need a formal context for objects like the
    sequence defined in equation (9). Clarity and formalism in these
    matters are especially important where random events generated by
    the bootstrap and by nature interact.}

  I have added a formal specification of the top-level probability
  space at the beginning of the mathematical appendix. You are
  right, this has made it easier to make precise statements in some
  of the results\,---\,e.g. the statement of Lemma~2 and the proof
  of Lemma~5. Thank you for this suggestion.  I will also point out
  that the paper defines several important sub-sigma-fields as they
  are necessary for the proofs, which gives additional clarity and
  formalism, and I have tried to improve in this respect in this
  version of the paper.

\end{enumerate}
\end{document}
